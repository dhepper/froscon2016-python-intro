% Generated by Sphinx.
\def\sphinxdocclass{report}
\documentclass[letterpaper,10pt,ngerman]{sphinxmanual}
\usepackage[utf8]{inputenc}
\DeclareUnicodeCharacter{00A0}{\nobreakspace}
\usepackage{cmap}
\usepackage[T1]{fontenc}
\usepackage{babel}
\usepackage{times}
\usepackage[Sonny]{fncychap}
\usepackage{longtable}
\usepackage{sphinx}
\usepackage{multirow}

\addto\captionsngerman{\renewcommand{\figurename}{Abb. }}
\addto\captionsngerman{\renewcommand{\tablename}{Tab. }}
\floatname{literal-block}{Quellcode }



\title{Python - eine Einführung Documentation}
\date{20.08.2016}
\release{2015.08.20}
\author{Daniel Hepper}
\newcommand{\sphinxlogo}{}
\renewcommand{\releasename}{Release}
\makeindex

\makeatletter
\def\PYG@reset{\let\PYG@it=\relax \let\PYG@bf=\relax%
    \let\PYG@ul=\relax \let\PYG@tc=\relax%
    \let\PYG@bc=\relax \let\PYG@ff=\relax}
\def\PYG@tok#1{\csname PYG@tok@#1\endcsname}
\def\PYG@toks#1+{\ifx\relax#1\empty\else%
    \PYG@tok{#1}\expandafter\PYG@toks\fi}
\def\PYG@do#1{\PYG@bc{\PYG@tc{\PYG@ul{%
    \PYG@it{\PYG@bf{\PYG@ff{#1}}}}}}}
\def\PYG#1#2{\PYG@reset\PYG@toks#1+\relax+\PYG@do{#2}}

\expandafter\def\csname PYG@tok@gd\endcsname{\def\PYG@tc##1{\textcolor[rgb]{0.63,0.00,0.00}{##1}}}
\expandafter\def\csname PYG@tok@gu\endcsname{\let\PYG@bf=\textbf\def\PYG@tc##1{\textcolor[rgb]{0.50,0.00,0.50}{##1}}}
\expandafter\def\csname PYG@tok@gt\endcsname{\def\PYG@tc##1{\textcolor[rgb]{0.00,0.27,0.87}{##1}}}
\expandafter\def\csname PYG@tok@gs\endcsname{\let\PYG@bf=\textbf}
\expandafter\def\csname PYG@tok@gr\endcsname{\def\PYG@tc##1{\textcolor[rgb]{1.00,0.00,0.00}{##1}}}
\expandafter\def\csname PYG@tok@cm\endcsname{\let\PYG@it=\textit\def\PYG@tc##1{\textcolor[rgb]{0.25,0.50,0.56}{##1}}}
\expandafter\def\csname PYG@tok@vg\endcsname{\def\PYG@tc##1{\textcolor[rgb]{0.73,0.38,0.84}{##1}}}
\expandafter\def\csname PYG@tok@m\endcsname{\def\PYG@tc##1{\textcolor[rgb]{0.13,0.50,0.31}{##1}}}
\expandafter\def\csname PYG@tok@mh\endcsname{\def\PYG@tc##1{\textcolor[rgb]{0.13,0.50,0.31}{##1}}}
\expandafter\def\csname PYG@tok@cs\endcsname{\def\PYG@tc##1{\textcolor[rgb]{0.25,0.50,0.56}{##1}}\def\PYG@bc##1{\setlength{\fboxsep}{0pt}\colorbox[rgb]{1.00,0.94,0.94}{\strut ##1}}}
\expandafter\def\csname PYG@tok@ge\endcsname{\let\PYG@it=\textit}
\expandafter\def\csname PYG@tok@vc\endcsname{\def\PYG@tc##1{\textcolor[rgb]{0.73,0.38,0.84}{##1}}}
\expandafter\def\csname PYG@tok@il\endcsname{\def\PYG@tc##1{\textcolor[rgb]{0.13,0.50,0.31}{##1}}}
\expandafter\def\csname PYG@tok@go\endcsname{\def\PYG@tc##1{\textcolor[rgb]{0.20,0.20,0.20}{##1}}}
\expandafter\def\csname PYG@tok@cp\endcsname{\def\PYG@tc##1{\textcolor[rgb]{0.00,0.44,0.13}{##1}}}
\expandafter\def\csname PYG@tok@gi\endcsname{\def\PYG@tc##1{\textcolor[rgb]{0.00,0.63,0.00}{##1}}}
\expandafter\def\csname PYG@tok@gh\endcsname{\let\PYG@bf=\textbf\def\PYG@tc##1{\textcolor[rgb]{0.00,0.00,0.50}{##1}}}
\expandafter\def\csname PYG@tok@ni\endcsname{\let\PYG@bf=\textbf\def\PYG@tc##1{\textcolor[rgb]{0.84,0.33,0.22}{##1}}}
\expandafter\def\csname PYG@tok@nl\endcsname{\let\PYG@bf=\textbf\def\PYG@tc##1{\textcolor[rgb]{0.00,0.13,0.44}{##1}}}
\expandafter\def\csname PYG@tok@nn\endcsname{\let\PYG@bf=\textbf\def\PYG@tc##1{\textcolor[rgb]{0.05,0.52,0.71}{##1}}}
\expandafter\def\csname PYG@tok@no\endcsname{\def\PYG@tc##1{\textcolor[rgb]{0.38,0.68,0.84}{##1}}}
\expandafter\def\csname PYG@tok@na\endcsname{\def\PYG@tc##1{\textcolor[rgb]{0.25,0.44,0.63}{##1}}}
\expandafter\def\csname PYG@tok@nb\endcsname{\def\PYG@tc##1{\textcolor[rgb]{0.00,0.44,0.13}{##1}}}
\expandafter\def\csname PYG@tok@nc\endcsname{\let\PYG@bf=\textbf\def\PYG@tc##1{\textcolor[rgb]{0.05,0.52,0.71}{##1}}}
\expandafter\def\csname PYG@tok@nd\endcsname{\let\PYG@bf=\textbf\def\PYG@tc##1{\textcolor[rgb]{0.33,0.33,0.33}{##1}}}
\expandafter\def\csname PYG@tok@ne\endcsname{\def\PYG@tc##1{\textcolor[rgb]{0.00,0.44,0.13}{##1}}}
\expandafter\def\csname PYG@tok@nf\endcsname{\def\PYG@tc##1{\textcolor[rgb]{0.02,0.16,0.49}{##1}}}
\expandafter\def\csname PYG@tok@si\endcsname{\let\PYG@it=\textit\def\PYG@tc##1{\textcolor[rgb]{0.44,0.63,0.82}{##1}}}
\expandafter\def\csname PYG@tok@s2\endcsname{\def\PYG@tc##1{\textcolor[rgb]{0.25,0.44,0.63}{##1}}}
\expandafter\def\csname PYG@tok@vi\endcsname{\def\PYG@tc##1{\textcolor[rgb]{0.73,0.38,0.84}{##1}}}
\expandafter\def\csname PYG@tok@nt\endcsname{\let\PYG@bf=\textbf\def\PYG@tc##1{\textcolor[rgb]{0.02,0.16,0.45}{##1}}}
\expandafter\def\csname PYG@tok@nv\endcsname{\def\PYG@tc##1{\textcolor[rgb]{0.73,0.38,0.84}{##1}}}
\expandafter\def\csname PYG@tok@s1\endcsname{\def\PYG@tc##1{\textcolor[rgb]{0.25,0.44,0.63}{##1}}}
\expandafter\def\csname PYG@tok@gp\endcsname{\let\PYG@bf=\textbf\def\PYG@tc##1{\textcolor[rgb]{0.78,0.36,0.04}{##1}}}
\expandafter\def\csname PYG@tok@sh\endcsname{\def\PYG@tc##1{\textcolor[rgb]{0.25,0.44,0.63}{##1}}}
\expandafter\def\csname PYG@tok@ow\endcsname{\let\PYG@bf=\textbf\def\PYG@tc##1{\textcolor[rgb]{0.00,0.44,0.13}{##1}}}
\expandafter\def\csname PYG@tok@sx\endcsname{\def\PYG@tc##1{\textcolor[rgb]{0.78,0.36,0.04}{##1}}}
\expandafter\def\csname PYG@tok@bp\endcsname{\def\PYG@tc##1{\textcolor[rgb]{0.00,0.44,0.13}{##1}}}
\expandafter\def\csname PYG@tok@c1\endcsname{\let\PYG@it=\textit\def\PYG@tc##1{\textcolor[rgb]{0.25,0.50,0.56}{##1}}}
\expandafter\def\csname PYG@tok@kc\endcsname{\let\PYG@bf=\textbf\def\PYG@tc##1{\textcolor[rgb]{0.00,0.44,0.13}{##1}}}
\expandafter\def\csname PYG@tok@c\endcsname{\let\PYG@it=\textit\def\PYG@tc##1{\textcolor[rgb]{0.25,0.50,0.56}{##1}}}
\expandafter\def\csname PYG@tok@mf\endcsname{\def\PYG@tc##1{\textcolor[rgb]{0.13,0.50,0.31}{##1}}}
\expandafter\def\csname PYG@tok@err\endcsname{\def\PYG@bc##1{\setlength{\fboxsep}{0pt}\fcolorbox[rgb]{1.00,0.00,0.00}{1,1,1}{\strut ##1}}}
\expandafter\def\csname PYG@tok@mb\endcsname{\def\PYG@tc##1{\textcolor[rgb]{0.13,0.50,0.31}{##1}}}
\expandafter\def\csname PYG@tok@ss\endcsname{\def\PYG@tc##1{\textcolor[rgb]{0.32,0.47,0.09}{##1}}}
\expandafter\def\csname PYG@tok@sr\endcsname{\def\PYG@tc##1{\textcolor[rgb]{0.14,0.33,0.53}{##1}}}
\expandafter\def\csname PYG@tok@mo\endcsname{\def\PYG@tc##1{\textcolor[rgb]{0.13,0.50,0.31}{##1}}}
\expandafter\def\csname PYG@tok@kd\endcsname{\let\PYG@bf=\textbf\def\PYG@tc##1{\textcolor[rgb]{0.00,0.44,0.13}{##1}}}
\expandafter\def\csname PYG@tok@mi\endcsname{\def\PYG@tc##1{\textcolor[rgb]{0.13,0.50,0.31}{##1}}}
\expandafter\def\csname PYG@tok@kn\endcsname{\let\PYG@bf=\textbf\def\PYG@tc##1{\textcolor[rgb]{0.00,0.44,0.13}{##1}}}
\expandafter\def\csname PYG@tok@o\endcsname{\def\PYG@tc##1{\textcolor[rgb]{0.40,0.40,0.40}{##1}}}
\expandafter\def\csname PYG@tok@kr\endcsname{\let\PYG@bf=\textbf\def\PYG@tc##1{\textcolor[rgb]{0.00,0.44,0.13}{##1}}}
\expandafter\def\csname PYG@tok@s\endcsname{\def\PYG@tc##1{\textcolor[rgb]{0.25,0.44,0.63}{##1}}}
\expandafter\def\csname PYG@tok@kp\endcsname{\def\PYG@tc##1{\textcolor[rgb]{0.00,0.44,0.13}{##1}}}
\expandafter\def\csname PYG@tok@w\endcsname{\def\PYG@tc##1{\textcolor[rgb]{0.73,0.73,0.73}{##1}}}
\expandafter\def\csname PYG@tok@kt\endcsname{\def\PYG@tc##1{\textcolor[rgb]{0.56,0.13,0.00}{##1}}}
\expandafter\def\csname PYG@tok@sc\endcsname{\def\PYG@tc##1{\textcolor[rgb]{0.25,0.44,0.63}{##1}}}
\expandafter\def\csname PYG@tok@sb\endcsname{\def\PYG@tc##1{\textcolor[rgb]{0.25,0.44,0.63}{##1}}}
\expandafter\def\csname PYG@tok@k\endcsname{\let\PYG@bf=\textbf\def\PYG@tc##1{\textcolor[rgb]{0.00,0.44,0.13}{##1}}}
\expandafter\def\csname PYG@tok@se\endcsname{\let\PYG@bf=\textbf\def\PYG@tc##1{\textcolor[rgb]{0.25,0.44,0.63}{##1}}}
\expandafter\def\csname PYG@tok@sd\endcsname{\let\PYG@it=\textit\def\PYG@tc##1{\textcolor[rgb]{0.25,0.44,0.63}{##1}}}

\def\PYGZbs{\char`\\}
\def\PYGZus{\char`\_}
\def\PYGZob{\char`\{}
\def\PYGZcb{\char`\}}
\def\PYGZca{\char`\^}
\def\PYGZam{\char`\&}
\def\PYGZlt{\char`\<}
\def\PYGZgt{\char`\>}
\def\PYGZsh{\char`\#}
\def\PYGZpc{\char`\%}
\def\PYGZdl{\char`\$}
\def\PYGZhy{\char`\-}
\def\PYGZsq{\char`\'}
\def\PYGZdq{\char`\"}
\def\PYGZti{\char`\~}
% for compatibility with earlier versions
\def\PYGZat{@}
\def\PYGZlb{[}
\def\PYGZrb{]}
\makeatother

\renewcommand\PYGZsq{\textquotesingle}

\begin{document}
\shorthandoff{"}
\maketitle
\tableofcontents
\phantomsection\label{index::doc}



\chapter{Agenda}
\label{index:einfuhrung-in-python}\label{index:agenda}\begin{itemize}
\item {} 
Was ist Python?

\item {} 
Warum Python?

\item {} 
Wie funktioniert Python?

\end{itemize}


\chapter{Was ist Python?}
\label{index:was-ist-python}\begin{itemize}
\item {} 
interpretiert

\item {} 
dynamisch typisiert

\item {} 
Multiparadigmensprache
\begin{itemize}
\item {} 
imperativ

\item {} 
strukturiert/prozedural

\item {} 
Objekt-orientiert

\item {} 
funktional

\item {} 
Aspekt-orientiert

\end{itemize}

\end{itemize}


\chapter{Geschichte}
\label{index:geschichte}\begin{itemize}
\item {} 
Entworfen 1991 von Guido van Rossum
\begin{itemize}
\item {} 
als Nachfolger von ABC

\item {} 
Monthy Python \textgreater{} Schlangen

\end{itemize}

\item {} 
aktuell Version 3.5.2 / 2.7.12
\begin{itemize}
\item {} 
3 \textgreater{} 2

\end{itemize}

\end{itemize}


\chapter{Warum Python?}
\label{index:warum-python}

\chapter{Sprache}
\label{index:sprache}\begin{itemize}
\item {} 
Fokus auf Lesbarkeit

\item {} 
einfach zu erlernen

\item {} 
ausdrucksstark und mächtig

\end{itemize}


\chapter{Batteries included}
\label{index:batteries-included}\begin{itemize}
\item {} 
Umfangreiche Standardbibliothek

\item {} 
gigantisches Open Source Ökosystem

\end{itemize}


\chapter{Gute Integrationsmöglichkeiten}
\label{index:gute-integrationsmoglichkeiten}\begin{itemize}
\item {} 
C

\item {} 
Java

\item {} 
.NET

\end{itemize}


\chapter{Einsatzbereiche}
\label{index:einsatzbereiche}\begin{itemize}
\item {} 
Skripte

\item {} 
Systemadministration

\item {} 
Kommandozeilentools

\item {} 
GUI-Programme

\item {} 
Web

\item {} 
3D

\item {} 
Scientific Computing: SciPy

\item {} 
Spiele

\item {} 
Micro-Controller

\item {} 
Bildung: OLPC, RaspberryPi

\item {} 
...

\end{itemize}


\chapter{Firmen \& Institutionen}
\label{index:firmen-institutionen}\begin{itemize}
\item {} 
Google, Instagram, Quora, Uber, Lieferheld...

\item {} 
Mozilla, Canonical, RedHat, Gentoo…

\item {} 
DLR, NASA, CERN...

\end{itemize}


\chapter{Community}
\label{index:community}\begin{itemize}
\item {} 
Mailinglisten

\item {} 
Usergroups

\item {} 
Konferenzen

\end{itemize}


\chapter{Wie funktioniert Python?}
\label{index:wie-funktioniert-python}

\chapter{Interpreter}
\label{index:interpreter}
\begin{Verbatim}[commandchars=\\\{\}]
\PYGZdl{} python3
Python 3.5.2 (v3.5.2:4def2a2901a5, Jun 26 2016, 10:47:25)
[GCC 4.2.1 (Apple Inc. build 5666) (dot 3)] on darwin
Type \PYGZdq{}help\PYGZdq{}, \PYGZdq{}copyright\PYGZdq{}, \PYGZdq{}credits\PYGZdq{} or \PYGZdq{}license\PYGZdq{} for more information.
\PYGZgt{}\PYGZgt{}\PYGZgt{} print(\PYGZsq{}Hello FrOSCon\PYGZsq{})
Hello FrOSCon
\PYGZgt{}\PYGZgt{}\PYGZgt{}
\end{Verbatim}


\chapter{Skript}
\label{index:skript}
\begin{Verbatim}[commandchars=\\\{\}]
\PYGZdl{} echo \PYGZdq{}print(\PYGZsq{}Hello FrOSCon\PYGZsq{})\PYGZdq{} \PYGZgt{} hello\PYGZus{}froscon.py
\PYGZdl{} python hello\PYGZus{}froscon.py
Hello FrOSCon
\PYGZdl{}
\end{Verbatim}


\chapter{Zeichenketten (Strings)}
\label{index:zeichenketten-strings}
\begin{Verbatim}[commandchars=\\\{\}]
\PYG{g+gp}{\PYGZgt{}\PYGZgt{}\PYGZgt{} }\PYG{l+s}{\PYGZsq{}}\PYG{l+s}{spam eggs}\PYG{l+s}{\PYGZsq{}}
\PYG{g+go}{\PYGZsq{}spam eggs\PYGZsq{}}
\end{Verbatim}

\begin{Verbatim}[commandchars=\\\{\}]
\PYG{g+gp}{\PYGZgt{}\PYGZgt{}\PYGZgt{} }\PYG{l+s}{\PYGZsq{}}\PYG{l+s}{doesn}\PYG{l+s+se}{\PYGZbs{}\PYGZsq{}}\PYG{l+s}{t}\PYG{l+s}{\PYGZsq{}}
\PYG{g+go}{\PYGZdq{}doesn\PYGZsq{}t\PYGZdq{}}
\end{Verbatim}

\begin{Verbatim}[commandchars=\\\{\}]
\PYG{g+gp}{\PYGZgt{}\PYGZgt{}\PYGZgt{} }\PYG{l+s}{\PYGZdq{}}\PYG{l+s}{doesn}\PYG{l+s}{\PYGZsq{}}\PYG{l+s}{t}\PYG{l+s}{\PYGZdq{}}
\PYG{g+go}{\PYGZdq{}doesn\PYGZsq{}t\PYGZdq{}}
\end{Verbatim}

\begin{Verbatim}[commandchars=\\\{\}]
\PYG{g+gp}{\PYGZgt{}\PYGZgt{}\PYGZgt{} }\PYG{l+s}{\PYGZsq{}}\PYG{l+s}{\PYGZdq{}}\PYG{l+s}{Ja,}\PYG{l+s}{\PYGZdq{}}\PYG{l+s}{, hat er gesagt.}\PYG{l+s}{\PYGZsq{}}
\PYG{g+go}{\PYGZsq{}\PYGZdq{}Ja,\PYGZdq{}, hat er gesagt.\PYGZsq{}}
\end{Verbatim}

\begin{Verbatim}[commandchars=\\\{\}]
\PYG{g+gp}{\PYGZgt{}\PYGZgt{}\PYGZgt{} }\PYG{l+s}{\PYGZdq{}}\PYG{l+s+se}{\PYGZbs{}\PYGZdq{}}\PYG{l+s}{Ja,}\PYG{l+s+se}{\PYGZbs{}\PYGZdq{}}\PYG{l+s}{, hat er gesagt.}\PYG{l+s}{\PYGZdq{}}
\PYG{g+go}{\PYGZsq{}\PYGZdq{}Ja,\PYGZdq{}, hat er gesagt.\PYGZsq{}}
\end{Verbatim}

\begin{Verbatim}[commandchars=\\\{\}]
\PYG{g+gp}{\PYGZgt{}\PYGZgt{}\PYGZgt{} }\PYG{l+s}{\PYGZsq{}}\PYG{l+s}{\PYGZdq{}}\PYG{l+s}{Isses nich}\PYG{l+s+se}{\PYGZbs{}\PYGZsq{}}\PYG{l+s}{,}\PYG{l+s}{\PYGZdq{}}\PYG{l+s}{, sagte sie.}\PYG{l+s}{\PYGZsq{}}
\PYG{g+go}{\PYGZsq{}\PYGZdq{}Isses nich\PYGZbs{}\PYGZsq{},\PYGZdq{}, sagte sie.}
\end{Verbatim}


\chapter{Lange Zeichenketten}
\label{index:lange-zeichenketten}
\begin{Verbatim}[commandchars=\\\{\}]
\PYG{g+gp}{\PYGZgt{}\PYGZgt{}\PYGZgt{} } \PYG{n}{hello} \PYG{o}{=} \PYG{l+s}{\PYGZdq{}}\PYG{l+s}{Dies ist eine ziemlich lange Zeichenkette,}\PYG{l+s+se}{\PYGZbs{}n}\PYG{l+s+se}{\PYGZbs{}}
\PYG{g+go}{.... die mehrere Zeilen Text enthält und wie man sie auch in C schreiben würde.\PYGZbs{}n\PYGZbs{}}
\PYG{g+go}{....    Achtung: Leerzeichen am Anfang haben eine Bedeutung\PYGZbs{}}
\PYG{g+go}{.... für die Darstellung.\PYGZdq{}}
\PYG{g+go}{.... print(hello)}
\end{Verbatim}

\begin{Verbatim}[commandchars=\\\{\}]
Dies ist eine ziemlich lange Zeichenkette,
die mehrere Zeilen Text enthält und wie man sie auch in C schreiben würde.
    Achtung: Leerzeichen am Anfang haben eine Bedeutung für die Darstellung.
\end{Verbatim}

\begin{Verbatim}[commandchars=\\\{\}]
\PYG{g+gp}{\PYGZgt{}\PYGZgt{}\PYGZgt{} } \PYG{n}{hello} \PYG{o}{=} \PYG{l+s}{\PYGZdq{}\PYGZdq{}\PYGZdq{}}\PYG{l+s}{Alternativ kann man auch drei Anführungszeichen verwenden,}
\PYG{g+go}{....            aber auch hier werden Leerzeichen beachtet.\PYGZdq{}\PYGZdq{}\PYGZdq{}}
\end{Verbatim}


\chapter{Unicode}
\label{index:unicode}
Seit Python 3.0 unterstützen durchgängig Unicode

\begin{Verbatim}[commandchars=\\\{\}]
\PYGZgt{}\PYGZgt{}\PYGZgt{} print(\PYGZdq{}Äpfel\PYGZdq{})
Äpfel
\PYGZgt{}\PYGZgt{}\PYGZgt{} ä = 3
\PYGZgt{}\PYGZgt{}\PYGZgt{} \PYGZdq{}Äpfel\PYGZdq{}.encode(\PYGZsq{}utf\PYGZhy{}8\PYGZsq{})
b\PYGZsq{}\PYGZbs{}xc3\PYGZbs{}x84pfel\PYGZsq{}
\end{Verbatim}


\chapter{Zahlen \& Arithmetik}
\label{index:zahlen-arithmetik}
\begin{Verbatim}[commandchars=\\\{\}]
\PYG{g+gp}{\PYGZgt{}\PYGZgt{}\PYGZgt{} }\PYG{l+m+mi}{2} \PYG{o}{+} \PYG{l+m+mi}{2}
\PYG{g+go}{4}
\PYG{g+gp}{\PYGZgt{}\PYGZgt{}\PYGZgt{} }\PYG{c}{\PYGZsh{} Dies ist ein Kommentar}
\PYG{g+gp}{... }\PYG{l+m+mi}{2} \PYG{o}{+} \PYG{l+m+mi}{2}
\PYG{g+go}{4}
\PYG{g+gp}{\PYGZgt{}\PYGZgt{}\PYGZgt{} }\PYG{l+m+mi}{2} \PYG{o}{+} \PYG{l+m+mi}{2}  \PYG{c}{\PYGZsh{} und dies ist ein Kommentar in derselben Zeile wie Code}
\PYG{g+go}{4}
\PYG{g+gp}{\PYGZgt{}\PYGZgt{}\PYGZgt{} }\PYG{p}{(}\PYG{l+m+mi}{50} \PYG{o}{\PYGZhy{}} \PYG{l+m+mi}{5} \PYG{o}{*} \PYG{l+m+mi}{6}\PYG{p}{)} \PYG{o}{/} \PYG{l+m+mi}{4}
\PYG{g+go}{5.0}
\PYG{g+gp}{\PYGZgt{}\PYGZgt{}\PYGZgt{} }\PYG{l+m+mi}{8} \PYG{o}{/} \PYG{l+m+mi}{5} \PYG{c}{\PYGZsh{} Brüche gehen nicht verloren, wenn man Ganzzahlen teilt}
\PYG{g+go}{1.6}
\end{Verbatim}


\chapter{Ganzzahldivision}
\label{index:ganzzahldivision}
\begin{Verbatim}[commandchars=\\\{\}]
\PYG{g+gp}{\PYGZgt{}\PYGZgt{}\PYGZgt{} }\PYG{c}{\PYGZsh{} Ganzzahldivision gibt ein abgerundetes Ergebnis zurück:}
\PYG{g+gp}{... }\PYG{l+m+mi}{7} \PYG{o}{/}\PYG{o}{/} \PYG{l+m+mi}{3}
\PYG{g+go}{2}
\PYG{g+gp}{\PYGZgt{}\PYGZgt{}\PYGZgt{} }\PYG{l+m+mi}{7} \PYG{o}{/}\PYG{o}{/} \PYG{o}{\PYGZhy{}}\PYG{l+m+mi}{3}
\PYG{g+go}{\PYGZhy{}3}
\end{Verbatim}


\chapter{Fließkommazahlen}
\label{index:flieszkommazahlen}
\begin{Verbatim}[commandchars=\\\{\}]
\PYG{g+gp}{\PYGZgt{}\PYGZgt{}\PYGZgt{} }\PYG{l+m+mi}{3} \PYG{o}{*} \PYG{l+m+mf}{3.75} \PYG{o}{/} \PYG{l+m+mf}{1.5}
\PYG{g+go}{7.5}
\PYG{g+gp}{\PYGZgt{}\PYGZgt{}\PYGZgt{} }\PYG{l+m+mf}{7.0} \PYG{o}{/} \PYG{l+m+mi}{2}
\PYG{g+go}{3.5}
\end{Verbatim}


\chapter{Komplexe Zahlen}
\label{index:komplexe-zahlen}
\begin{Verbatim}[commandchars=\\\{\}]
\PYG{g+gp}{\PYGZgt{}\PYGZgt{}\PYGZgt{} }\PYG{l+m+mi}{1j}
\PYG{g+go}{1j}
\PYG{g+gp}{\PYGZgt{}\PYGZgt{}\PYGZgt{} }\PYG{l+m+mi}{1j} \PYG{o}{*} \PYG{l+m+mi}{1}\PYG{n}{J}
\PYG{g+go}{(\PYGZhy{}1+0j)}
\PYG{g+gp}{\PYGZgt{}\PYGZgt{}\PYGZgt{} }\PYG{l+m+mi}{1j} \PYG{o}{*} \PYG{n+nb}{complex}\PYG{p}{(}\PYG{l+m+mi}{0}\PYG{p}{,} \PYG{l+m+mi}{1}\PYG{p}{)}
\PYG{g+go}{(\PYGZhy{}1+0j)}
\PYG{g+gp}{\PYGZgt{}\PYGZgt{}\PYGZgt{} }\PYG{l+m+mi}{3} \PYG{o}{+} \PYG{l+m+mi}{1j} \PYG{o}{*} \PYG{l+m+mi}{3}
\PYG{g+go}{(3+3j)}
\PYG{g+gp}{\PYGZgt{}\PYGZgt{}\PYGZgt{} }\PYG{p}{(}\PYG{l+m+mi}{3} \PYG{o}{+} \PYG{l+m+mi}{1j}\PYG{p}{)} \PYG{o}{*} \PYG{l+m+mi}{3}
\PYG{g+go}{(9+3j)}
\PYG{g+gp}{\PYGZgt{}\PYGZgt{}\PYGZgt{} }\PYG{p}{(}\PYG{l+m+mi}{1} \PYG{o}{+} \PYG{l+m+mi}{2j}\PYG{p}{)} \PYG{o}{/} \PYG{p}{(}\PYG{l+m+mi}{1} \PYG{o}{+} \PYG{l+m+mi}{1j}\PYG{p}{)}
\PYG{g+go}{(1.5+0.5j)}
\end{Verbatim}


\chapter{Variablen}
\label{index:variablen}
\begin{Verbatim}[commandchars=\\\{\}]
\PYG{g+gp}{\PYGZgt{}\PYGZgt{}\PYGZgt{} }\PYG{n}{width} \PYG{o}{=} \PYG{l+m+mi}{20}
\PYG{g+gp}{\PYGZgt{}\PYGZgt{}\PYGZgt{} }\PYG{n}{height} \PYG{o}{=} \PYG{l+m+mi}{5} \PYG{o}{*} \PYG{l+m+mi}{9}
\PYG{g+gp}{\PYGZgt{}\PYGZgt{}\PYGZgt{} }\PYG{n}{width} \PYG{o}{*} \PYG{n}{height}
\PYG{g+go}{900}
\end{Verbatim}


\chapter{Unser erster Fehler}
\label{index:unser-erster-fehler}
\begin{Verbatim}[commandchars=\\\{\}]
\PYG{g+gp}{\PYGZgt{}\PYGZgt{}\PYGZgt{} }\PYG{c}{\PYGZsh{} Versuche eine undefinierte Variable abzurufen}
\PYG{g+gp}{... }\PYG{n}{n}
\PYG{g+gt}{Traceback (most recent call last):}
\PYG{g+gr}{ File \PYGZdq{}\PYGZlt{}stdin\PYGZgt{}\PYGZdq{}, line 1, in \PYGZlt{}module\PYGZgt{}}
\PYG{g+gr}{NameError}: \PYG{n}{name \PYGZsq{}n\PYGZsq{} is not defined}
\end{Verbatim}


\chapter{Mehrfachzuweisung}
\label{index:mehrfachzuweisung}
\begin{Verbatim}[commandchars=\\\{\}]
\PYG{g+gp}{\PYGZgt{}\PYGZgt{}\PYGZgt{} }\PYG{n}{x} \PYG{o}{=} \PYG{n}{y} \PYG{o}{=} \PYG{n}{z} \PYG{o}{=} \PYG{l+m+mi}{0}  \PYG{c}{\PYGZsh{} Null für x, y und z}
\PYG{g+gp}{\PYGZgt{}\PYGZgt{}\PYGZgt{} }\PYG{n}{x}
\PYG{g+go}{0}
\PYG{g+gp}{\PYGZgt{}\PYGZgt{}\PYGZgt{} }\PYG{n}{y}
\PYG{g+go}{0}
\PYG{g+gp}{\PYGZgt{}\PYGZgt{}\PYGZgt{} }\PYG{n}{z}
\PYG{g+go}{0}
\end{Verbatim}


\chapter{Addition und Multiplikation von Zeichenketten}
\label{index:addition-und-multiplikation-von-zeichenketten}
\begin{Verbatim}[commandchars=\\\{\}]
\PYG{g+gp}{\PYGZgt{}\PYGZgt{}\PYGZgt{} }\PYG{n}{word} \PYG{o}{=} \PYG{l+s}{\PYGZsq{}}\PYG{l+s}{Help}\PYG{l+s}{\PYGZsq{}} \PYG{o}{+} \PYG{l+s}{\PYGZsq{}}\PYG{l+s}{A}\PYG{l+s}{\PYGZsq{}}
\PYG{g+gp}{\PYGZgt{}\PYGZgt{}\PYGZgt{} }\PYG{n}{word}
\PYG{g+go}{\PYGZsq{}HelpA\PYGZsq{}}
\PYG{g+gp}{\PYGZgt{}\PYGZgt{}\PYGZgt{} }\PYG{l+s}{\PYGZsq{}}\PYG{l+s}{\PYGZlt{}}\PYG{l+s}{\PYGZsq{}} \PYG{o}{+} \PYG{n}{word}\PYG{o}{*}\PYG{l+m+mi}{5} \PYG{o}{+} \PYG{l+s}{\PYGZsq{}}\PYG{l+s}{\PYGZgt{}}\PYG{l+s}{\PYGZsq{}}
\PYG{g+go}{\PYGZsq{}\PYGZlt{}HelpAHelpAHelpAHelpAHelpA\PYGZgt{}\PYGZsq{}}
\end{Verbatim}


\chapter{Indizierung}
\label{index:indizierung}
\begin{Verbatim}[commandchars=\\\{\}]
\PYG{g+gp}{\PYGZgt{}\PYGZgt{}\PYGZgt{} }\PYG{n}{word}
\PYG{g+go}{\PYGZsq{}HelpA\PYGZsq{}}
\PYG{g+gp}{\PYGZgt{}\PYGZgt{}\PYGZgt{} }\PYG{n}{word}\PYG{p}{[}\PYG{l+m+mi}{4}\PYG{p}{]}
\PYG{g+go}{\PYGZsq{}A\PYGZsq{}}
\PYG{g+gp}{\PYGZgt{}\PYGZgt{}\PYGZgt{} }\PYG{n}{word}\PYG{p}{[}\PYG{l+m+mi}{2}\PYG{p}{]}
\PYG{g+go}{\PYGZsq{}r\PYGZsq{}}
\end{Verbatim}


\chapter{Slices}
\label{index:slices}
\begin{Verbatim}[commandchars=\\\{\}]
\PYG{g+gp}{\PYGZgt{}\PYGZgt{}\PYGZgt{} }\PYG{n}{word}\PYG{p}{[}\PYG{l+m+mi}{0}\PYG{p}{:}\PYG{l+m+mi}{2}\PYG{p}{]}
\PYG{g+go}{\PYGZsq{}He\PYGZsq{}}
\PYG{g+gp}{\PYGZgt{}\PYGZgt{}\PYGZgt{} }\PYG{n}{word}\PYG{p}{[}\PYG{l+m+mi}{2}\PYG{p}{:}\PYG{l+m+mi}{4}\PYG{p}{]}
\PYG{g+go}{\PYGZsq{}lp\PYGZsq{}}
\end{Verbatim}

\begin{Verbatim}[commandchars=\\\{\}]
\PYG{g+gp}{\PYGZgt{}\PYGZgt{}\PYGZgt{} }\PYG{n}{word}\PYG{p}{[}\PYG{p}{:}\PYG{l+m+mi}{2}\PYG{p}{]}
\PYG{g+go}{\PYGZsq{}He\PYGZsq{}}
\PYG{g+gp}{\PYGZgt{}\PYGZgt{}\PYGZgt{} }\PYG{n}{word}\PYG{p}{[}\PYG{l+m+mi}{2}\PYG{p}{:}\PYG{p}{]}
\PYG{g+go}{\PYGZsq{}lpA\PYGZsq{}}
\PYG{g+gp}{\PYGZgt{}\PYGZgt{}\PYGZgt{} }\PYG{n}{word}\PYG{p}{[}\PYG{p}{:}\PYG{p}{]}
\end{Verbatim}


\chapter{Zeichenketten ändern}
\label{index:zeichenketten-andern}
\begin{Verbatim}[commandchars=\\\{\}]
\PYG{g+gp}{\PYGZgt{}\PYGZgt{}\PYGZgt{} }\PYG{n}{word}\PYG{p}{[}\PYG{l+m+mi}{0}\PYG{p}{]} \PYG{o}{=} \PYG{l+s}{\PYGZsq{}}\PYG{l+s}{x}\PYG{l+s}{\PYGZsq{}}
\PYG{g+gt}{Traceback (most recent call last):}
\PYG{g+gr}{ File \PYGZdq{}\PYGZlt{}stdin\PYGZgt{}\PYGZdq{}, line 1, in ?}
\PYG{g+gr}{TypeError}: \PYG{n}{\PYGZsq{}str\PYGZsq{} object does not support item assignment}
\PYG{g+gp}{\PYGZgt{}\PYGZgt{}\PYGZgt{} }\PYG{n}{word}\PYG{p}{[}\PYG{p}{:}\PYG{l+m+mi}{1}\PYG{p}{]} \PYG{o}{=} \PYG{l+s}{\PYGZsq{}}\PYG{l+s}{Splat}\PYG{l+s}{\PYGZsq{}}
\PYG{g+gt}{Traceback (most recent call last):}
\PYG{g+gr}{ File \PYGZdq{}\PYGZlt{}stdin\PYGZgt{}\PYGZdq{}, line 1, in ?}
\PYG{g+gr}{TypeError}: \PYG{n}{\PYGZsq{}str\PYGZsq{} object does not support slice assignment}
\end{Verbatim}

\begin{Verbatim}[commandchars=\\\{\}]
\PYG{g+gp}{\PYGZgt{}\PYGZgt{}\PYGZgt{} }\PYG{l+s}{\PYGZsq{}}\PYG{l+s}{x}\PYG{l+s}{\PYGZsq{}} \PYG{o}{+} \PYG{n}{word}\PYG{p}{[}\PYG{l+m+mi}{1}\PYG{p}{:}\PYG{p}{]}
\PYG{g+go}{\PYGZsq{}xelpA\PYGZsq{}}
\PYG{g+gp}{\PYGZgt{}\PYGZgt{}\PYGZgt{} }\PYG{l+s}{\PYGZsq{}}\PYG{l+s}{Splat}\PYG{l+s}{\PYGZsq{}} \PYG{o}{+} \PYG{n}{word}\PYG{p}{[}\PYG{l+m+mi}{4}\PYG{p}{]}
\PYG{g+go}{\PYGZsq{}SplatA\PYGZsq{}}
\end{Verbatim}


\chapter{Negative Indices}
\label{index:negative-indices}
\begin{Verbatim}[commandchars=\\\{\}]
\PYG{g+gp}{\PYGZgt{}\PYGZgt{}\PYGZgt{} }\PYG{n}{word}\PYG{p}{[}\PYG{o}{\PYGZhy{}}\PYG{l+m+mi}{1}\PYG{p}{]}     \PYG{c}{\PYGZsh{} Das letzte Zeichen}
\PYG{g+go}{\PYGZsq{}A\PYGZsq{}}
\PYG{g+gp}{\PYGZgt{}\PYGZgt{}\PYGZgt{} }\PYG{n}{word}\PYG{p}{[}\PYG{o}{\PYGZhy{}}\PYG{l+m+mi}{2}\PYG{p}{]}     \PYG{c}{\PYGZsh{} Das vorletzte Zeichen}
\PYG{g+go}{\PYGZsq{}p\PYGZsq{}}
\PYG{g+gp}{\PYGZgt{}\PYGZgt{}\PYGZgt{} }\PYG{n}{word}\PYG{p}{[}\PYG{o}{\PYGZhy{}}\PYG{l+m+mi}{2}\PYG{p}{:}\PYG{p}{]}    \PYG{c}{\PYGZsh{} Die letzten zwei Zeichen}
\PYG{g+go}{\PYGZsq{}pA\PYGZsq{}}
\PYG{g+gp}{\PYGZgt{}\PYGZgt{}\PYGZgt{} }\PYG{n}{word}\PYG{p}{[}\PYG{p}{:}\PYG{o}{\PYGZhy{}}\PYG{l+m+mi}{2}\PYG{p}{]}    \PYG{c}{\PYGZsh{} Alles außer den letzten beiden Zeichen}
\PYG{g+go}{\PYGZsq{}Hel\PYGZsq{}}
\end{Verbatim}

Achtung: -0 ist dasselbe wie 0.

\begin{Verbatim}[commandchars=\\\{\}]
\PYG{g+gp}{\PYGZgt{}\PYGZgt{}\PYGZgt{} }\PYG{n}{word}\PYG{p}{[}\PYG{o}{\PYGZhy{}}\PYG{l+m+mi}{0}\PYG{p}{]}     \PYG{c}{\PYGZsh{} (da \PYGZhy{}0 gleich 0)}
\PYG{g+go}{\PYGZsq{}H\PYGZsq{}}
\end{Verbatim}

\begin{notice}{note}{Bemerkung:}
Indizes können auch negative Zahlen sein --- dann wird von rechts nach links
gezählt. Zum Beispiel
\end{notice}


\chapter{Fehler beim Zugriff}
\label{index:fehler-beim-zugriff}
\begin{Verbatim}[commandchars=\\\{\}]
\PYG{g+gp}{\PYGZgt{}\PYGZgt{}\PYGZgt{} }\PYG{n}{word}\PYG{p}{[}\PYG{o}{\PYGZhy{}}\PYG{l+m+mi}{100}\PYG{p}{:}\PYG{p}{]}
\PYG{g+go}{\PYGZsq{}HelpA\PYGZsq{}}
\PYG{g+gp}{\PYGZgt{}\PYGZgt{}\PYGZgt{} }\PYG{n}{word}\PYG{p}{[}\PYG{o}{\PYGZhy{}}\PYG{l+m+mi}{10}\PYG{p}{]}    \PYG{c}{\PYGZsh{} Fehler}
\PYG{g+gt}{Traceback (most recent call last):}
\PYG{g+gr}{ File \PYGZdq{}\PYGZlt{}stdin\PYGZgt{}\PYGZdq{}, line 1, in ?}
\PYG{g+gr}{IndexError}: \PYG{n}{string index out of range}
\end{Verbatim}


\chapter{Listen}
\label{index:listen}
\begin{Verbatim}[commandchars=\\\{\}]
\PYG{g+gp}{\PYGZgt{}\PYGZgt{}\PYGZgt{} }\PYG{n}{a} \PYG{o}{=} \PYG{p}{[}\PYG{l+s}{\PYGZsq{}}\PYG{l+s}{spam}\PYG{l+s}{\PYGZsq{}}\PYG{p}{,} \PYG{l+s}{\PYGZsq{}}\PYG{l+s}{eggs}\PYG{l+s}{\PYGZsq{}}\PYG{p}{,} \PYG{l+m+mi}{100}\PYG{p}{,} \PYG{l+m+mi}{1234}\PYG{p}{]}
\PYG{g+gp}{\PYGZgt{}\PYGZgt{}\PYGZgt{} }\PYG{n}{a}
\PYG{g+go}{[\PYGZsq{}spam\PYGZsq{}, \PYGZsq{}eggs\PYGZsq{}, 100, 1234]}
\end{Verbatim}


\chapter{Listen}
\label{index:id1}
\begin{Verbatim}[commandchars=\\\{\}]
\PYG{g+gp}{\PYGZgt{}\PYGZgt{}\PYGZgt{} }\PYG{n}{a}\PYG{p}{[}\PYG{l+m+mi}{0}\PYG{p}{]}
\PYG{g+go}{\PYGZsq{}spam\PYGZsq{}}
\PYG{g+gp}{\PYGZgt{}\PYGZgt{}\PYGZgt{} }\PYG{n}{a}\PYG{p}{[}\PYG{l+m+mi}{3}\PYG{p}{]}
\PYG{g+go}{1234}
\end{Verbatim}

\begin{Verbatim}[commandchars=\\\{\}]
\PYG{g+gp}{\PYGZgt{}\PYGZgt{}\PYGZgt{} }\PYG{n}{a}\PYG{p}{[}\PYG{o}{\PYGZhy{}}\PYG{l+m+mi}{2}\PYG{p}{]}
\PYG{g+go}{100}
\PYG{g+gp}{\PYGZgt{}\PYGZgt{}\PYGZgt{} }\PYG{n}{a}\PYG{p}{[}\PYG{l+m+mi}{1}\PYG{p}{:}\PYG{o}{\PYGZhy{}}\PYG{l+m+mi}{1}\PYG{p}{]}
\PYG{g+go}{[\PYGZsq{}eggs\PYGZsq{}, 100]}
\PYG{g+gp}{\PYGZgt{}\PYGZgt{}\PYGZgt{} }\PYG{n}{a}\PYG{p}{[}\PYG{p}{:}\PYG{l+m+mi}{2}\PYG{p}{]} \PYG{o}{+} \PYG{p}{[}\PYG{l+s}{\PYGZsq{}}\PYG{l+s}{bacon}\PYG{l+s}{\PYGZsq{}}\PYG{p}{,} \PYG{l+m+mi}{2}\PYG{o}{*}\PYG{l+m+mi}{2}\PYG{p}{]}
\PYG{g+go}{[\PYGZsq{}spam\PYGZsq{}, \PYGZsq{}eggs\PYGZsq{}, \PYGZsq{}bacon\PYGZsq{}, 4]}
\PYG{g+gp}{\PYGZgt{}\PYGZgt{}\PYGZgt{} }\PYG{l+m+mi}{3}\PYG{o}{*}\PYG{n}{a}\PYG{p}{[}\PYG{p}{:}\PYG{l+m+mi}{3}\PYG{p}{]} \PYG{o}{+} \PYG{p}{[}\PYG{l+s}{\PYGZsq{}}\PYG{l+s}{Boo!}\PYG{l+s}{\PYGZsq{}}\PYG{p}{]}
\PYG{g+go}{[\PYGZsq{}spam\PYGZsq{}, \PYGZsq{}eggs\PYGZsq{}, 100, \PYGZsq{}spam\PYGZsq{}, \PYGZsq{}eggs\PYGZsq{}, 100, \PYGZsq{}spam\PYGZsq{}, \PYGZsq{}eggs\PYGZsq{}, 100, \PYGZsq{}Boo!\PYGZsq{}]}
\end{Verbatim}


\chapter{Listen}
\label{index:id2}
\begin{Verbatim}[commandchars=\\\{\}]
\PYG{g+gp}{\PYGZgt{}\PYGZgt{}\PYGZgt{} }\PYG{n}{a}
\PYG{g+go}{[\PYGZsq{}spam\PYGZsq{}, \PYGZsq{}eggs\PYGZsq{}, 100, 1234]}
\PYG{g+gp}{\PYGZgt{}\PYGZgt{}\PYGZgt{} }\PYG{n}{a}\PYG{p}{[}\PYG{l+m+mi}{2}\PYG{p}{]} \PYG{o}{=} \PYG{n}{a}\PYG{p}{[}\PYG{l+m+mi}{2}\PYG{p}{]} \PYG{o}{+} \PYG{l+m+mi}{23}
\PYG{g+gp}{\PYGZgt{}\PYGZgt{}\PYGZgt{} }\PYG{n}{a}
\PYG{g+go}{[\PYGZsq{}spam\PYGZsq{}, \PYGZsq{}eggs\PYGZsq{}, 123, 1234]}
\end{Verbatim}


\chapter{Listen}
\label{index:id3}
\begin{Verbatim}[commandchars=\\\{\}]
\PYG{g+gp}{\PYGZgt{}\PYGZgt{}\PYGZgt{} }\PYG{c}{\PYGZsh{} Ein paar Elemente ersetzen:}
\PYG{g+gp}{... }\PYG{n}{a}\PYG{p}{[}\PYG{l+m+mi}{0}\PYG{p}{:}\PYG{l+m+mi}{2}\PYG{p}{]} \PYG{o}{=} \PYG{p}{[}\PYG{l+m+mi}{1}\PYG{p}{,} \PYG{l+m+mi}{12}\PYG{p}{]}
\PYG{g+gp}{\PYGZgt{}\PYGZgt{}\PYGZgt{} }\PYG{n}{a}
\PYG{g+go}{[1, 12, 123, 1234]}
\PYG{g+gp}{\PYGZgt{}\PYGZgt{}\PYGZgt{} }\PYG{c}{\PYGZsh{} Ein paar entfernen:}
\PYG{g+gp}{... }\PYG{n}{a}\PYG{p}{[}\PYG{l+m+mi}{0}\PYG{p}{:}\PYG{l+m+mi}{2}\PYG{p}{]} \PYG{o}{=} \PYG{p}{[}\PYG{p}{]}
\PYG{g+gp}{\PYGZgt{}\PYGZgt{}\PYGZgt{} }\PYG{n}{a}
\PYG{g+go}{[123, 1234]}
\PYG{g+gp}{\PYGZgt{}\PYGZgt{}\PYGZgt{} }\PYG{c}{\PYGZsh{} Ein paar einfügen:}
\PYG{g+gp}{... }\PYG{n}{a}\PYG{p}{[}\PYG{l+m+mi}{1}\PYG{p}{:}\PYG{l+m+mi}{1}\PYG{p}{]} \PYG{o}{=} \PYG{p}{[}\PYG{l+s}{\PYGZsq{}}\PYG{l+s}{bletch}\PYG{l+s}{\PYGZsq{}}\PYG{p}{,} \PYG{l+s}{\PYGZsq{}}\PYG{l+s}{xyzzy}\PYG{l+s}{\PYGZsq{}}\PYG{p}{]}
\PYG{g+gp}{\PYGZgt{}\PYGZgt{}\PYGZgt{} }\PYG{n}{a}
\PYG{g+go}{[123, \PYGZsq{}bletch\PYGZsq{}, \PYGZsq{}xyzzy\PYGZsq{}, 1234]}
\PYG{g+gp}{\PYGZgt{}\PYGZgt{}\PYGZgt{} }\PYG{c}{\PYGZsh{} (Eine Kopie von) sich selbst am Anfang einfügen:}
\PYG{g+gp}{\PYGZgt{}\PYGZgt{}\PYGZgt{} }\PYG{n}{a}\PYG{p}{[}\PYG{p}{:}\PYG{l+m+mi}{0}\PYG{p}{]} \PYG{o}{=} \PYG{n}{a}
\PYG{g+gp}{\PYGZgt{}\PYGZgt{}\PYGZgt{} }\PYG{n}{a}
\PYG{g+go}{[123, \PYGZsq{}bletch\PYGZsq{}, \PYGZsq{}xyzzy\PYGZsq{}, 1234, 123, \PYGZsq{}bletch\PYGZsq{}, \PYGZsq{}xyzzy\PYGZsq{}, 1234]}
\PYG{g+gp}{\PYGZgt{}\PYGZgt{}\PYGZgt{} }\PYG{c}{\PYGZsh{} Die Liste leeren: Alle Elemente durch eine leere Liste  ersetzen}
\PYG{g+gp}{\PYGZgt{}\PYGZgt{}\PYGZgt{} }\PYG{n}{a}\PYG{p}{[}\PYG{p}{:}\PYG{p}{]} \PYG{o}{=} \PYG{p}{[}\PYG{p}{]}
\PYG{g+gp}{\PYGZgt{}\PYGZgt{}\PYGZgt{} }\PYG{n}{a}
\PYG{g+go}{[]}
\end{Verbatim}


\chapter{Länge einer Liste}
\label{index:lange-einer-liste}
\begin{Verbatim}[commandchars=\\\{\}]
\PYG{g+gp}{\PYGZgt{}\PYGZgt{}\PYGZgt{} }\PYG{n}{a} \PYG{o}{=} \PYG{p}{[}\PYG{l+s}{\PYGZsq{}}\PYG{l+s}{a}\PYG{l+s}{\PYGZsq{}}\PYG{p}{,} \PYG{l+s}{\PYGZsq{}}\PYG{l+s}{b}\PYG{l+s}{\PYGZsq{}}\PYG{p}{,} \PYG{l+s}{\PYGZsq{}}\PYG{l+s}{c}\PYG{l+s}{\PYGZsq{}}\PYG{p}{,} \PYG{l+s}{\PYGZsq{}}\PYG{l+s}{d}\PYG{l+s}{\PYGZsq{}}\PYG{p}{]}
\PYG{g+gp}{\PYGZgt{}\PYGZgt{}\PYGZgt{} }\PYG{n+nb}{len}\PYG{p}{(}\PYG{n}{a}\PYG{p}{)}
\PYG{g+go}{4}
\end{Verbatim}


\chapter{Listen verschachteln}
\label{index:listen-verschachteln}
\begin{Verbatim}[commandchars=\\\{\}]
\PYG{g+gp}{\PYGZgt{}\PYGZgt{}\PYGZgt{} }\PYG{n}{q} \PYG{o}{=} \PYG{p}{[}\PYG{l+m+mi}{2}\PYG{p}{,} \PYG{l+m+mi}{3}\PYG{p}{]}
\PYG{g+gp}{\PYGZgt{}\PYGZgt{}\PYGZgt{} }\PYG{n}{p} \PYG{o}{=} \PYG{p}{[}\PYG{l+m+mi}{1}\PYG{p}{,} \PYG{n}{q}\PYG{p}{,} \PYG{l+m+mi}{4}\PYG{p}{]}
\PYG{g+gp}{\PYGZgt{}\PYGZgt{}\PYGZgt{} }\PYG{n+nb}{len}\PYG{p}{(}\PYG{n}{p}\PYG{p}{)}
\PYG{g+go}{3}
\PYG{g+gp}{\PYGZgt{}\PYGZgt{}\PYGZgt{} }\PYG{n}{p}\PYG{p}{[}\PYG{l+m+mi}{1}\PYG{p}{]}
\PYG{g+go}{[2, 3]}
\PYG{g+gp}{\PYGZgt{}\PYGZgt{}\PYGZgt{} }\PYG{n}{p}\PYG{p}{[}\PYG{l+m+mi}{1}\PYG{p}{]}\PYG{p}{[}\PYG{l+m+mi}{0}\PYG{p}{]}
\PYG{g+go}{2}
\end{Verbatim}

\begin{Verbatim}[commandchars=\\\{\}]
\PYG{g+gp}{\PYGZgt{}\PYGZgt{}\PYGZgt{} }\PYG{n}{p}\PYG{p}{[}\PYG{l+m+mi}{1}\PYG{p}{]}\PYG{o}{.}\PYG{n}{append}\PYG{p}{(}\PYG{l+s}{\PYGZsq{}}\PYG{l+s}{xtra}\PYG{l+s}{\PYGZsq{}}\PYG{p}{)}
\PYG{g+gp}{\PYGZgt{}\PYGZgt{}\PYGZgt{} }\PYG{n}{p}
\PYG{g+go}{[1, [2, 3, \PYGZsq{}xtra\PYGZsq{}], 4]}
\PYG{g+gp}{\PYGZgt{}\PYGZgt{}\PYGZgt{} }\PYG{n}{q}
\PYG{g+go}{[2, 3, \PYGZsq{}xtra\PYGZsq{}]}
\end{Verbatim}

\begin{notice}{note}{Bemerkung:}
Beachte, dass im letzten Beispiel \code{p{[}1{]}} und \code{q} wirklich auf dasselbe
Objekt zeigen!
\begin{itemize}
\item {} 
Schüssel =\textgreater{} Wert

\item {} 
auch bekannt als
\begin{itemize}
\item {} 
``assoziativer Speicher''

\item {} 
``assoziative Arrays''

\item {} 
Hash

\item {} 
Map

\end{itemize}

\item {} 
jedes nicht veränderbares Objekt kann ein Schlüssel sein

\end{itemize}

\begin{notice}{note}{Bemerkung:}
Keine Listen
\end{notice}

\begin{Verbatim}[commandchars=\\\{\}]
\PYG{g+gp}{\PYGZgt{}\PYGZgt{}\PYGZgt{} }\PYG{n}{tel} \PYG{o}{=} \PYG{p}{\PYGZob{}}\PYG{l+s}{\PYGZsq{}}\PYG{l+s}{jack}\PYG{l+s}{\PYGZsq{}}\PYG{p}{:} \PYG{l+m+mi}{4098}\PYG{p}{,} \PYG{l+s}{\PYGZsq{}}\PYG{l+s}{sape}\PYG{l+s}{\PYGZsq{}}\PYG{p}{:} \PYG{l+m+mi}{4139}\PYG{p}{\PYGZcb{}}
\PYG{g+gp}{\PYGZgt{}\PYGZgt{}\PYGZgt{} }\PYG{n}{tel}\PYG{p}{[}\PYG{l+s}{\PYGZsq{}}\PYG{l+s}{guido}\PYG{l+s}{\PYGZsq{}}\PYG{p}{]} \PYG{o}{=} \PYG{l+m+mi}{4127}
\PYG{g+gp}{\PYGZgt{}\PYGZgt{}\PYGZgt{} }\PYG{n}{tel}
\PYG{g+go}{\PYGZob{}\PYGZsq{}sape\PYGZsq{}: 4139, \PYGZsq{}guido\PYGZsq{}: 4127, \PYGZsq{}jack\PYGZsq{}: 4098\PYGZcb{}}
\PYG{g+gp}{\PYGZgt{}\PYGZgt{}\PYGZgt{} }\PYG{n}{tel}\PYG{p}{[}\PYG{l+s}{\PYGZsq{}}\PYG{l+s}{jack}\PYG{l+s}{\PYGZsq{}}\PYG{p}{]}
\PYG{g+go}{4098}
\PYG{g+gp}{\PYGZgt{}\PYGZgt{}\PYGZgt{} }\PYG{k}{del} \PYG{n}{tel}\PYG{p}{[}\PYG{l+s}{\PYGZsq{}}\PYG{l+s}{sape}\PYG{l+s}{\PYGZsq{}}\PYG{p}{]}
\PYG{g+gp}{\PYGZgt{}\PYGZgt{}\PYGZgt{} }\PYG{n}{tel}\PYG{p}{[}\PYG{l+s}{\PYGZsq{}}\PYG{l+s}{irv}\PYG{l+s}{\PYGZsq{}}\PYG{p}{]} \PYG{o}{=} \PYG{l+m+mi}{4127}
\PYG{g+gp}{\PYGZgt{}\PYGZgt{}\PYGZgt{} }\PYG{n}{tel}
\PYG{g+go}{\PYGZob{}\PYGZsq{}guido\PYGZsq{}: 4127, \PYGZsq{}irv\PYGZsq{}: 4127, \PYGZsq{}jack\PYGZsq{}: 4098\PYGZcb{}}
\PYG{g+gp}{\PYGZgt{}\PYGZgt{}\PYGZgt{} }\PYG{n+nb}{list}\PYG{p}{(}\PYG{n}{tel}\PYG{o}{.}\PYG{n}{keys}\PYG{p}{(}\PYG{p}{)}\PYG{p}{)}
\PYG{g+go}{[\PYGZsq{}irv\PYGZsq{}, \PYGZsq{}guido\PYGZsq{}, \PYGZsq{}jack\PYGZsq{}]}
\PYG{g+gp}{\PYGZgt{}\PYGZgt{}\PYGZgt{} }\PYG{n+nb}{sorted}\PYG{p}{(}\PYG{n}{tel}\PYG{o}{.}\PYG{n}{keys}\PYG{p}{(}\PYG{p}{)}\PYG{p}{)}
\PYG{g+go}{[\PYGZsq{}guido\PYGZsq{}, \PYGZsq{}irv\PYGZsq{}, \PYGZsq{}jack\PYGZsq{}]}
\PYG{g+gp}{\PYGZgt{}\PYGZgt{}\PYGZgt{} }\PYG{l+s}{\PYGZsq{}}\PYG{l+s}{guido}\PYG{l+s}{\PYGZsq{}} \PYG{o+ow}{in} \PYG{n}{tel}
\PYG{g+go}{True}
\PYG{g+gp}{\PYGZgt{}\PYGZgt{}\PYGZgt{} }\PYG{l+s}{\PYGZsq{}}\PYG{l+s}{jack}\PYG{l+s}{\PYGZsq{}} \PYG{o+ow}{not} \PYG{o+ow}{in} \PYG{n}{tel}
\PYG{g+go}{False}
\end{Verbatim}

Dictionaries direkt aus Sequenzen erstellen...

\begin{Verbatim}[commandchars=\\\{\}]
\PYG{g+gp}{\PYGZgt{}\PYGZgt{}\PYGZgt{} }\PYG{n+nb}{dict}\PYG{p}{(}\PYG{p}{[}\PYG{p}{(}\PYG{l+s}{\PYGZsq{}}\PYG{l+s}{sape}\PYG{l+s}{\PYGZsq{}}\PYG{p}{,} \PYG{l+m+mi}{4139}\PYG{p}{)}\PYG{p}{,} \PYG{p}{(}\PYG{l+s}{\PYGZsq{}}\PYG{l+s}{guido}\PYG{l+s}{\PYGZsq{}}\PYG{p}{,} \PYG{l+m+mi}{4127}\PYG{p}{)}\PYG{p}{,} \PYG{p}{(}\PYG{l+s}{\PYGZsq{}}\PYG{l+s}{jack}\PYG{l+s}{\PYGZsq{}}\PYG{p}{,} \PYG{l+m+mi}{4098}\PYG{p}{)}\PYG{p}{]}\PYG{p}{)}
\PYG{g+go}{\PYGZob{}\PYGZsq{}sape\PYGZsq{}: 4139, \PYGZsq{}jack\PYGZsq{}: 4098, \PYGZsq{}guido\PYGZsq{}: 4127\PYGZcb{}}
\end{Verbatim}

...oder aus Keyword-Argumenten

\begin{Verbatim}[commandchars=\\\{\}]
\PYG{g+gp}{\PYGZgt{}\PYGZgt{}\PYGZgt{} }\PYG{n+nb}{dict}\PYG{p}{(}\PYG{n}{sape}\PYG{o}{=}\PYG{l+m+mi}{4139}\PYG{p}{,} \PYG{n}{guido}\PYG{o}{=}\PYG{l+m+mi}{4127}\PYG{p}{,} \PYG{n}{jack}\PYG{o}{=}\PYG{l+m+mi}{4098}\PYG{p}{)}
\PYG{g+go}{\PYGZob{}\PYGZsq{}sape\PYGZsq{}: 4139, \PYGZsq{}jack\PYGZsq{}: 4098, \PYGZsq{}guido\PYGZsq{}: 4127\PYGZcb{}}
\end{Verbatim}
\end{notice}


\chapter{Erste Schritte zur Programmierung}
\label{index:erste-schritte-zur-programmierung}
\begin{Verbatim}[commandchars=\\\{\}]
\PYG{g+gp}{\PYGZgt{}\PYGZgt{}\PYGZgt{} }\PYG{c}{\PYGZsh{} Fibonacci\PYGZhy{}Folge:}
\PYG{g+gp}{... }\PYG{c}{\PYGZsh{} Die Summe der letzten beiden Elemente ergibt das nächste}
\PYG{g+gp}{... }\PYG{n}{a}\PYG{p}{,} \PYG{n}{b} \PYG{o}{=} \PYG{l+m+mi}{0}\PYG{p}{,} \PYG{l+m+mi}{1}
\PYG{g+gp}{\PYGZgt{}\PYGZgt{}\PYGZgt{} }\PYG{k}{while} \PYG{n}{b} \PYG{o}{\PYGZlt{}} \PYG{l+m+mi}{10}\PYG{p}{:}
\PYG{g+gp}{... }    \PYG{k}{print}\PYG{p}{(}\PYG{n}{b}\PYG{p}{)}
\PYG{g+gp}{... }    \PYG{n}{a}\PYG{p}{,} \PYG{n}{b} \PYG{o}{=} \PYG{n}{b}\PYG{p}{,} \PYG{n}{a}\PYG{o}{+}\PYG{n}{b}
\PYG{g+gp}{...}
\PYG{g+go}{1}
\PYG{g+go}{1}
\PYG{g+go}{2}
\PYG{g+go}{3}
\PYG{g+go}{5}
\PYG{g+go}{8}
\end{Verbatim}


\chapter{\texttt{if}-Anweisungen}
\label{index:if-anweisungen}
Ein Beispiel zur \href{https://docs.python.org/reference/compound\_stmts.html\#if}{\code{if}}-Anweisung

\begin{Verbatim}[commandchars=\\\{\}]
\PYG{g+gp}{\PYGZgt{}\PYGZgt{}\PYGZgt{} }\PYG{n}{x} \PYG{o}{=} \PYG{n+nb}{int}\PYG{p}{(}\PYG{n+nb}{input}\PYG{p}{(}\PYG{l+s}{\PYGZdq{}}\PYG{l+s}{Please enter an integer: }\PYG{l+s}{\PYGZdq{}}\PYG{p}{)}\PYG{p}{)}
\PYG{g+go}{Please enter an integer: 42}
\PYG{g+gp}{\PYGZgt{}\PYGZgt{}\PYGZgt{} }\PYG{k}{if} \PYG{n}{x} \PYG{o}{\PYGZlt{}} \PYG{l+m+mi}{0}\PYG{p}{:}
\PYG{g+gp}{... }     \PYG{n}{x} \PYG{o}{=} \PYG{l+m+mi}{0}
\PYG{g+gp}{... }     \PYG{k}{print}\PYG{p}{(}\PYG{l+s}{\PYGZsq{}}\PYG{l+s}{Negative changed to zero}\PYG{l+s}{\PYGZsq{}}\PYG{p}{)}
\PYG{g+gp}{... }\PYG{k}{elif} \PYG{n}{x} \PYG{o}{==} \PYG{l+m+mi}{0}\PYG{p}{:}
\PYG{g+gp}{... }     \PYG{k}{print}\PYG{p}{(}\PYG{l+s}{\PYGZsq{}}\PYG{l+s}{Zero}\PYG{l+s}{\PYGZsq{}}\PYG{p}{)}
\PYG{g+gp}{... }\PYG{k}{elif} \PYG{n}{x} \PYG{o}{==} \PYG{l+m+mi}{1}\PYG{p}{:}
\PYG{g+gp}{... }     \PYG{k}{print}\PYG{p}{(}\PYG{l+s}{\PYGZsq{}}\PYG{l+s}{Single}\PYG{l+s}{\PYGZsq{}}\PYG{p}{)}
\PYG{g+gp}{... }\PYG{k}{else}\PYG{p}{:}
\PYG{g+gp}{... }     \PYG{k}{print}\PYG{p}{(}\PYG{l+s}{\PYGZsq{}}\PYG{l+s}{More}\PYG{l+s}{\PYGZsq{}}\PYG{p}{)}
\PYG{g+gp}{...}
\PYG{g+go}{More}
\end{Verbatim}

\begin{notice}{note}{Bemerkung:}
\href{https://docs.python.org/reference/compound\_stmts.html\#else}{\code{else}}-Zweig oder \href{https://docs.python.org/reference/compound\_stmts.html\#elif}{\code{elif}}-Zweige sind optional. Im Unterschied
zum \href{https://docs.python.org/reference/compound\_stmts.html\#else}{\code{else}}-Zweig, der nur einmal vorkommen kann, ist eine Abfolge von
mehreren \href{https://docs.python.org/reference/compound\_stmts.html\#elif}{\code{elif}}-Zweigen möglich; dadurch lassen sich verschachtelte
Einrückungen vermeiden.  Eine Abfolge von \href{https://docs.python.org/reference/compound\_stmts.html\#if}{\code{if}} ... \href{https://docs.python.org/reference/compound\_stmts.html\#elif}{\code{elif}} ...
\href{https://docs.python.org/reference/compound\_stmts.html\#elif}{\code{elif}}-Zweigen ersetzt die \code{switch}- oder \code{case}-Konstrukte anderer
Programmiersprachen.
\end{notice}


\chapter{\texttt{for}-Anweisungen}
\label{index:for-anweisungen}
\begin{Verbatim}[commandchars=\\\{\}]
\PYG{g+gp}{\PYGZgt{}\PYGZgt{}\PYGZgt{} }\PYG{c}{\PYGZsh{} Die Längen einiger Zeichenketten ermitteln:}
\PYG{g+gp}{... }\PYG{n}{a} \PYG{o}{=} \PYG{p}{[}\PYG{l+s}{\PYGZsq{}}\PYG{l+s}{Katze}\PYG{l+s}{\PYGZsq{}}\PYG{p}{,} \PYG{l+s}{\PYGZsq{}}\PYG{l+s}{Fenster}\PYG{l+s}{\PYGZsq{}}\PYG{p}{,} \PYG{l+s}{\PYGZsq{}}\PYG{l+s}{rauswerfen}\PYG{l+s}{\PYGZsq{}}\PYG{p}{]}
\PYG{g+gp}{\PYGZgt{}\PYGZgt{}\PYGZgt{} }\PYG{k}{for} \PYG{n}{x} \PYG{o+ow}{in} \PYG{n}{a}\PYG{p}{:}
\PYG{g+gp}{... }    \PYG{k}{print}\PYG{p}{(}\PYG{n}{x}\PYG{p}{,} \PYG{n+nb}{len}\PYG{p}{(}\PYG{n}{x}\PYG{p}{)}\PYG{p}{)}
\PYG{g+gp}{...}
\PYG{g+go}{Katze 5}
\PYG{g+go}{Fenster 7}
\PYG{g+go}{rauswerfen 10}
\end{Verbatim}

\begin{notice}{note}{Bemerkung:}
Die \href{https://docs.python.org/reference/compound\_stmts.html\#for}{\code{for}}-Anweisung in Python unterscheidet sich ein wenig von der, die
man von C oder Pascal her kennt. Man kann nicht nur über eine Zahlenfolge
iterieren (wie in Pascal) oder lediglich Schrittweite und Abbruchbedingung
festlegen (wie in C), sondern über eine beliebige Sequenz (also z. B. eine Liste
oder Zeichenkette), und zwar in der Reihenfolge, in der die Elemente in der
Sequenz vorkommen. Zum Beispiel:
\end{notice}


\chapter{Die Funktion \texttt{range()}}
\label{index:die-funktion-range}
\begin{Verbatim}[commandchars=\\\{\}]
\PYG{g+gp}{\PYGZgt{}\PYGZgt{}\PYGZgt{} }\PYG{k}{for} \PYG{n}{i} \PYG{o+ow}{in} \PYG{n+nb}{range}\PYG{p}{(}\PYG{l+m+mi}{5}\PYG{p}{)}\PYG{p}{:}
\PYG{g+gp}{... }    \PYG{k}{print}\PYG{p}{(}\PYG{n}{i}\PYG{p}{)}
\PYG{g+gp}{...}
\PYG{g+go}{0}
\PYG{g+go}{1}
\PYG{g+go}{2}
\PYG{g+go}{3}
\PYG{g+go}{4}
\end{Verbatim}

\begin{Verbatim}[commandchars=\\\{\}]
\PYG{g+gp}{\PYGZgt{}\PYGZgt{}\PYGZgt{} }\PYG{k}{print}\PYG{p}{(}\PYG{n+nb}{range}\PYG{p}{(}\PYG{l+m+mi}{10}\PYG{p}{)}\PYG{p}{)}
\PYG{g+go}{range(0, 10)}
\end{Verbatim}

\begin{Verbatim}[commandchars=\\\{\}]
\PYG{g+gp}{\PYGZgt{}\PYGZgt{}\PYGZgt{} }\PYG{n+nb}{list}\PYG{p}{(}\PYG{n+nb}{range}\PYG{p}{(}\PYG{l+m+mi}{5}\PYG{p}{)}\PYG{p}{)}
\PYG{g+go}{[0, 1, 2, 3, 4]}
\end{Verbatim}


\chapter{\texttt{break}, \texttt{continue} und \texttt{else} bei Schleifen}
\label{index:break-continue-und-else-bei-schleifen}
\begin{Verbatim}[commandchars=\\\{\}]
\PYG{g+gp}{\PYGZgt{}\PYGZgt{}\PYGZgt{} }\PYG{k}{for} \PYG{n}{n} \PYG{o+ow}{in} \PYG{n+nb}{range}\PYG{p}{(}\PYG{l+m+mi}{2}\PYG{p}{,} \PYG{l+m+mi}{10}\PYG{p}{)}\PYG{p}{:}
\PYG{g+gp}{... }    \PYG{k}{for} \PYG{n}{x} \PYG{o+ow}{in} \PYG{n+nb}{range}\PYG{p}{(}\PYG{l+m+mi}{2}\PYG{p}{,} \PYG{n}{n}\PYG{p}{)}\PYG{p}{:}
\PYG{g+gp}{... }        \PYG{k}{if} \PYG{n}{n} \PYG{o}{\PYGZpc{}} \PYG{n}{x} \PYG{o}{==} \PYG{l+m+mi}{0}\PYG{p}{:}
\PYG{g+gp}{... }            \PYG{k}{print}\PYG{p}{(}\PYG{n}{n}\PYG{p}{,} \PYG{l+s}{\PYGZsq{}}\PYG{l+s}{equals}\PYG{l+s}{\PYGZsq{}}\PYG{p}{,} \PYG{n}{x}\PYG{p}{,} \PYG{l+s}{\PYGZsq{}}\PYG{l+s}{*}\PYG{l+s}{\PYGZsq{}}\PYG{p}{,} \PYG{n}{n}\PYG{o}{/}\PYG{o}{/}\PYG{n}{x}\PYG{p}{)}
\PYG{g+gp}{... }            \PYG{k}{break}
\PYG{g+gp}{... }    \PYG{k}{else}\PYG{p}{:}
\PYG{g+gp}{... }        \PYG{c}{\PYGZsh{} Schleife wurde durchlaufen, ohne dass ein Faktor gefunden wurde}
\PYG{g+gp}{... }        \PYG{k}{print}\PYG{p}{(}\PYG{n}{n}\PYG{p}{,} \PYG{l+s}{\PYGZsq{}}\PYG{l+s}{is a prime number}\PYG{l+s}{\PYGZsq{}}\PYG{p}{)}
\PYG{g+gp}{...}
\PYG{g+go}{2 is a prime number}
\PYG{g+go}{3 is a prime number}
\PYG{g+go}{4 equals 2 * 2}
\PYG{g+go}{5 is a prime number}
\PYG{g+go}{6 equals 2 * 3}
\PYG{g+go}{7 is a prime number}
\PYG{g+go}{8 equals 2 * 4}
\PYG{g+go}{9 equals 3 * 3}
\end{Verbatim}


\chapter{\texttt{pass}-Anweisungen}
\label{index:pass-anweisungen}
\begin{Verbatim}[commandchars=\\\{\}]
\PYG{g+gp}{\PYGZgt{}\PYGZgt{}\PYGZgt{} }\PYG{k}{while} \PYG{n+nb+bp}{True}\PYG{p}{:}
\PYG{g+gp}{... }    \PYG{k}{pass}  \PYG{c}{\PYGZsh{} geschäftiges Warten auf den Tastatur\PYGZhy{}Interrupt (Strg+C)}
\PYG{g+gp}{...}
\end{Verbatim}

\begin{Verbatim}[commandchars=\\\{\}]
\PYG{g+gp}{\PYGZgt{}\PYGZgt{}\PYGZgt{} }\PYG{k}{class} \PYG{n+nc}{MyEmptyClass}\PYG{p}{:}
\PYG{g+gp}{... }    \PYG{k}{pass}
\PYG{g+gp}{...}
\end{Verbatim}

\begin{Verbatim}[commandchars=\\\{\}]
\PYG{g+gp}{\PYGZgt{}\PYGZgt{}\PYGZgt{} }\PYG{k}{def} \PYG{n+nf}{initlog}\PYG{p}{(}\PYG{o}{*}\PYG{n}{args}\PYG{p}{)}\PYG{p}{:}
\PYG{g+gp}{... }    \PYG{k}{pass}   \PYG{c}{\PYGZsh{} Implementieren nicht vergessen!}
\PYG{g+gp}{...}
\end{Verbatim}


\chapter{Funktionen definieren}
\label{index:funktionen-definieren}
\begin{Verbatim}[commandchars=\\\{\}]
\PYG{g+gp}{\PYGZgt{}\PYGZgt{}\PYGZgt{} }\PYG{k}{def} \PYG{n+nf}{fib}\PYG{p}{(}\PYG{n}{n}\PYG{p}{)}\PYG{p}{:}    \PYG{c}{\PYGZsh{} die Fibonacci\PYGZhy{}Folge bis n ausgeben}
\PYG{g+gp}{... }    \PYG{l+s+sd}{\PYGZdq{}\PYGZdq{}\PYGZdq{}Print the Fibonacci series up to n.\PYGZdq{}\PYGZdq{}\PYGZdq{}}
\PYG{g+gp}{... }    \PYG{n}{a}\PYG{p}{,} \PYG{n}{b} \PYG{o}{=} \PYG{l+m+mi}{0}\PYG{p}{,} \PYG{l+m+mi}{1}
\PYG{g+gp}{... }    \PYG{k}{while} \PYG{n}{a} \PYG{o}{\PYGZlt{}} \PYG{n}{n}\PYG{p}{:}
\PYG{g+gp}{... }        \PYG{k}{print}\PYG{p}{(}\PYG{n}{a}\PYG{p}{,} \PYG{n}{end}\PYG{o}{=}\PYG{l+s}{\PYGZsq{}}\PYG{l+s}{ }\PYG{l+s}{\PYGZsq{}}\PYG{p}{)}
\PYG{g+gp}{... }        \PYG{n}{a}\PYG{p}{,} \PYG{n}{b} \PYG{o}{=} \PYG{n}{b}\PYG{p}{,} \PYG{n}{a}\PYG{o}{+}\PYG{n}{b}
\PYG{g+gp}{... }    \PYG{k}{print}\PYG{p}{(}\PYG{p}{)}
\PYG{g+gp}{...}
\end{Verbatim}

\begin{Verbatim}[commandchars=\\\{\}]
\PYG{g+gp}{\PYGZgt{}\PYGZgt{}\PYGZgt{} }\PYG{n}{fib}\PYG{p}{(}\PYG{l+m+mi}{2000}\PYG{p}{)}
\PYG{g+go}{0 1 1 2 3 5 8 13 21 34 55 89 144 233 377 610 987 1597}
\end{Verbatim}

\begin{notice}{note}{Bemerkung:}
Das Schlüsselwort \href{https://docs.python.org/reference/compound\_stmts.html\#def}{\code{def}} leitet die \emph{Definition} einer Funktion ein.
Darauf folgt der Funktionsname und eine Auflistung der formalen Parameter, die
allerdings auch leer sein kann. Die Anweisungen, die den Funktionskörper bilden,
beginnen in der nächsten Zeile und müssen eingerückt sein.

Die erste Anweisung des Funktionskörpers kann auch ein Zeichenkettenliteral
sein, ein so genannter Dokumentationsstring der Funktion, auch \emph{Docstring}
genannt. (Mehr zu Docstrings kann im Abschnitt \href{https://docs.python.org/tutorial/controlflow.html\#tut-docstrings}{\DUspan{}{Documentation Strings}}
nachgelesen werden.) Es gibt Werkzeuge, die Docstrings verwenden, um automatisch
Online-Dokumentation oder gedruckte Dokumentation zu erzeugen oder es dem
Anwender ermöglichen, interaktiv den Code zu durchsuchen. Die Verwendung von
Docstrings ist eine gute Konvention, an die man sich bei der Programmierung nach
Möglichkeit halten sollte.

Beim \emph{Aufruf} einer Funktion kommt es zur Bildung eines lokalen Namensraums, der
sich auf alle Bezeichner erstreckt, die im Funktionsrumpf (durch Zuweisung oder
als Elemente der Parameterliste) neu definiert werden. Diese Bezeichner werden
mit den ihnen zugeordneten Objekten in einer lokalen Symboltabelle abgelegt.

Wenn im Funktionsrumpf ein Bezeichner vorkommt, wird der Name zunächst in der
lokalen Symboltabelle gesucht, danach in den lokalen Symboltabellen der
umgebenden Funktionen, dann in der globalen Symboltabelle und schließlich in der
Symboltabelle der eingebauten Namen. Darum ist es ohne weiteres nicht möglich,
einer globalen Variablen innerhalb des lokalen Namensraums einer Funktion einen
Wert zuzuweisen.  Dadurch würde stattdessen eine neue, namensgleiche lokale
Variable definiert, die die namensgleiche globale Variable überdeckt und dadurch
auch den lesenden Zugriff auf diese globale Variable verhindert. Ein lesender
Zugriff auf globale Variablen ist ansonsten immer möglich, ein schreibender
Zugriff nur unter Verwendung der \href{https://docs.python.org/reference/simple\_stmts.html\#global}{\code{global}}-Anweisung.

Die konkreten Parameter (Argumente), die beim Funktionsaufruf übergeben werden,
werden den formalen Parametern der Parameterliste zugeordnet und gehören damit
zur lokalen Symboltabelle der Funktion. Das heißt, Argumente werden über \emph{call
by value} übergeben (wobei der \emph{Wert} allerdings immer eine \emph{Referenz} auf ein
Objekt ist, nicht der Wert des Objektes selbst) {\color{red}\bfseries{}{[}\#{]}\_}. Wenn eine Funktion eine
andere Funktion aufruft, wird eine neue lokale Symboltabelle für diesen Aufruf
erzeugt.

Eine Funktionsdefinition fügt den Funktionsnamen in die lokale Symboltabelle
ein. Der Wert des Funktionsnamens hat einen Typ, der vom Interpreter als
benutzerdefinierte Funktion erkannt wird. Dieser Wert kann einem anderen Namen
zugewiesen werden, der dann ebenfalls als Funktion genutzt werden kann und so
als Möglichkeit zur Umbenennung dient.

\begin{Verbatim}[commandchars=\\\{\}]
\PYG{g+gp}{\PYGZgt{}\PYGZgt{}\PYGZgt{} }\PYG{n}{fib}
\PYG{g+go}{\PYGZlt{}function fib at 10042ed0\PYGZgt{}}
\PYG{g+gp}{\PYGZgt{}\PYGZgt{}\PYGZgt{} }\PYG{n}{f} \PYG{o}{=} \PYG{n}{fib}
\PYG{g+gp}{\PYGZgt{}\PYGZgt{}\PYGZgt{} }\PYG{n}{f}\PYG{p}{(}\PYG{l+m+mi}{100}\PYG{p}{)}
\PYG{g+go}{0 2 1 2 3 5 8 13 21 34 55 89}
\end{Verbatim}

Wer Erfahrung mit anderen Programmiersprachen hat, wird vielleicht einwenden,
dass \code{fib} gar keine Funktion, sondern eine Prozedur ist, da sie keinen Wert
zurückgibt.  Tatsächlich geben aber auch Funktionen \emph{ohne} eine
\href{https://docs.python.org/reference/simple\_stmts.html\#return}{\code{return}}-Anweisung einen Wert zurück, wenn auch einen eher
langweiligen, nämlich den eingebauten Namen \code{None} (``nichts'').  Die Ausgabe
des Wertes \code{None} wird normalerweise vom Interpreter unterdrückt, wenn es der
einzige Wert wäre, der ausgegeben wird. Möchte man ihn sehen, kann man ihn
mittels \href{https://docs.python.org/library/functions.html\#print}{\code{print()}} sichtbar machen.:
\end{notice}


\chapter{Prozeduren vs. Funktionen}
\label{index:prozeduren-vs-funktionen}
\begin{Verbatim}[commandchars=\\\{\}]
\PYG{g+gp}{\PYGZgt{}\PYGZgt{}\PYGZgt{} }\PYG{n}{fib}\PYG{p}{(}\PYG{l+m+mi}{0}\PYG{p}{)}
\PYG{g+gp}{\PYGZgt{}\PYGZgt{}\PYGZgt{} }\PYG{k}{print}\PYG{p}{(}\PYG{n}{fib}\PYG{p}{(}\PYG{l+m+mi}{0}\PYG{p}{)}\PYG{p}{)}
\PYG{g+go}{None}
\end{Verbatim}


\chapter{Funktionen}
\label{index:funktionen}
\begin{Verbatim}[commandchars=\\\{\}]
\PYG{g+gp}{\PYGZgt{}\PYGZgt{}\PYGZgt{} }\PYG{k}{def} \PYG{n+nf}{fib2}\PYG{p}{(}\PYG{n}{n}\PYG{p}{)}\PYG{p}{:}
\PYG{g+gp}{... }    \PYG{l+s+sd}{\PYGZdq{}\PYGZdq{}\PYGZdq{}Return a list containing the Fibonacci series up to n.\PYGZdq{}\PYGZdq{}\PYGZdq{}}
\PYG{g+gp}{... }    \PYG{n}{result} \PYG{o}{=} \PYG{n+nb}{list}\PYG{p}{(}\PYG{p}{)}
\PYG{g+gp}{... }    \PYG{n}{a}\PYG{p}{,} \PYG{n}{b} \PYG{o}{=} \PYG{l+m+mi}{0}\PYG{p}{,} \PYG{l+m+mi}{1}
\PYG{g+gp}{... }    \PYG{k}{while} \PYG{n}{a} \PYG{o}{\PYGZlt{}} \PYG{n}{n}\PYG{p}{:}
\PYG{g+gp}{... }        \PYG{n}{result}\PYG{o}{.}\PYG{n}{append}\PYG{p}{(}\PYG{n}{a}\PYG{p}{)}
\PYG{g+gp}{... }        \PYG{n}{a}\PYG{p}{,} \PYG{n}{b} \PYG{o}{=} \PYG{n}{b}\PYG{p}{,} \PYG{n}{a} \PYG{o}{+} \PYG{n}{b}
\PYG{g+gp}{... }    \PYG{k}{return} \PYG{n}{result}
\PYG{g+gp}{...}
\PYG{g+gp}{\PYGZgt{}\PYGZgt{}\PYGZgt{} }\PYG{n}{f100} \PYG{o}{=} \PYG{n}{fib2}\PYG{p}{(}\PYG{l+m+mi}{100}\PYG{p}{)}
\PYG{g+gp}{\PYGZgt{}\PYGZgt{}\PYGZgt{} }\PYG{n}{f100}
\PYG{g+go}{[0, 1, 1, 2, 3, 5, 8, 13, 21, 34, 55, 89]}
\end{Verbatim}

\begin{notice}{note}{Bemerkung:}
Dieses Beispiel zeigt einige neue Eigenschaften von Python:
\begin{itemize}
\item {} 
Die \href{https://docs.python.org/reference/simple\_stmts.html\#return}{\code{return}}-Anweisung gibt einen Wert von einer Funktion
zurück. Ohne einen Ausdruck als Argument gibt \href{https://docs.python.org/reference/simple\_stmts.html\#return}{\code{return}} \code{None}
zurück; das gleiche gilt, wenn eine \href{https://docs.python.org/reference/simple\_stmts.html\#return}{\code{return}}-Anweisung fehlt.

\item {} 
Die Anweisung \code{result.append(a)} ruft eine \emph{Methode} des Listenobjektes in
\code{result} auf. Eine Methode ist eine Funktion, die zu einem Objekt `gehört'
und wird mittels Punktnotation (\code{obj.methodname}) dargestellt. Dabei ist
\code{obj} irgendein Objekt (es kann auch ein Ausdruck sein) und \code{methodname}
der Name einer Methode, die vom Typ des Objektes definiert wird.
Unterschiedliche Typen definieren verschiedene Methoden. Methoden
verschiedener Typen können denselben Namen haben ohne doppeldeutig zu sein.
(Es ist auch möglich, eigene Objekttypen zu erstellen, indem man \emph{Klassen}
benutzt, siehe \href{https://docs.python.org/tutorial/classes.html\#tut-classes}{\DUspan{}{Classes}}.) Die Methode \code{append()}, die im
Beispiel gezeigt wird, ist für Listenobjekte definiert. Sie hängt ein neues
Element an das Ende der Liste an. Im Beispiel ist es äquivalent zu \code{result
= result + {[}a{]}}, aber effizienter.

\end{itemize}
\end{notice}


\chapter{Standardwerte für Argumente}
\label{index:tut-defaultargs}\label{index:standardwerte-fur-argumente}
\begin{Verbatim}[commandchars=\\\{\}]
\PYG{k}{def} \PYG{n+nf}{ask\PYGZus{}ok}\PYG{p}{(}\PYG{n}{prompt}\PYG{p}{,} \PYG{n}{retries}\PYG{o}{=}\PYG{l+m+mi}{4}\PYG{p}{,} \PYG{n}{complaint}\PYG{o}{=}\PYG{l+s}{\PYGZsq{}}\PYG{l+s}{Bitte Ja oder Nein!}\PYG{l+s}{\PYGZsq{}}\PYG{p}{)}\PYG{p}{:}
   \PYG{k}{while} \PYG{n+nb+bp}{True}\PYG{p}{:}
       \PYG{n}{ok} \PYG{o}{=} \PYG{n+nb}{input}\PYG{p}{(}\PYG{n}{prompt}\PYG{p}{)}
       \PYG{k}{if} \PYG{n}{ok} \PYG{o+ow}{in} \PYG{p}{(}\PYG{l+s}{\PYGZsq{}}\PYG{l+s}{j}\PYG{l+s}{\PYGZsq{}}\PYG{p}{,} \PYG{l+s}{\PYGZsq{}}\PYG{l+s}{J}\PYG{l+s}{\PYGZsq{}}\PYG{p}{,} \PYG{l+s}{\PYGZsq{}}\PYG{l+s}{ja}\PYG{l+s}{\PYGZsq{}}\PYG{p}{,} \PYG{l+s}{\PYGZsq{}}\PYG{l+s}{Ja}\PYG{l+s}{\PYGZsq{}}\PYG{p}{)}\PYG{p}{:} \PYG{k}{return} \PYG{n+nb+bp}{True}
       \PYG{k}{if} \PYG{n}{ok} \PYG{o+ow}{in} \PYG{p}{(}\PYG{l+s}{\PYGZsq{}}\PYG{l+s}{n}\PYG{l+s}{\PYGZsq{}}\PYG{p}{,} \PYG{l+s}{\PYGZsq{}}\PYG{l+s}{N}\PYG{l+s}{\PYGZsq{}}\PYG{p}{,} \PYG{l+s}{\PYGZsq{}}\PYG{l+s}{ne}\PYG{l+s}{\PYGZsq{}}\PYG{p}{,} \PYG{l+s}{\PYGZsq{}}\PYG{l+s}{Ne}\PYG{l+s}{\PYGZsq{}}\PYG{p}{,} \PYG{l+s}{\PYGZsq{}}\PYG{l+s}{Nein}\PYG{l+s}{\PYGZsq{}}\PYG{p}{)}\PYG{p}{:} \PYG{k}{return} \PYG{n+nb+bp}{False}
       \PYG{n}{retries} \PYG{o}{=} \PYG{n}{retries} \PYG{o}{\PYGZhy{}} \PYG{l+m+mi}{1}
       \PYG{k}{if} \PYG{n}{retries} \PYG{o}{\PYGZlt{}} \PYG{l+m+mi}{0}\PYG{p}{:}
           \PYG{k}{raise} \PYG{n+ne}{IOError}\PYG{p}{(}\PYG{l+s}{\PYGZsq{}}\PYG{l+s}{Benutzer abgelehnt!}\PYG{l+s}{\PYGZsq{}}\PYG{p}{)}
       \PYG{k}{print}\PYG{p}{(}\PYG{n}{complaint}\PYG{p}{)}
\end{Verbatim}
\begin{itemize}
\item {} 
\code{ask\_ok("Willst du wirklich aufhören?")}

\item {} 
\code{ask\_ok("Willst du die Datei überschreiben?", 2)}

\item {} 
\code{ask\_ok("Willst du die Datei überschreiben?", 2, "Komm schon, nur Ja oder
Nein")}

\end{itemize}


\chapter{Keyword-Argumente}
\label{index:keyword-argumente}
\begin{Verbatim}[commandchars=\\\{\}]
\PYG{k}{def} \PYG{n+nf}{ask\PYGZus{}ok}\PYG{p}{(}\PYG{n}{prompt}\PYG{p}{,} \PYG{n}{retries}\PYG{o}{=}\PYG{l+m+mi}{4}\PYG{p}{,} \PYG{n}{complaint}\PYG{o}{=}\PYG{l+s}{\PYGZsq{}}\PYG{l+s}{Bitte Ja oder Nein!}\PYG{l+s}{\PYGZsq{}}\PYG{p}{)}\PYG{p}{:}
   \PYG{k}{while} \PYG{n+nb+bp}{True}\PYG{p}{:}
       \PYG{n}{ok} \PYG{o}{=} \PYG{n+nb}{input}\PYG{p}{(}\PYG{n}{prompt}\PYG{p}{)}
       \PYG{k}{if} \PYG{n}{ok} \PYG{o+ow}{in} \PYG{p}{(}\PYG{l+s}{\PYGZsq{}}\PYG{l+s}{j}\PYG{l+s}{\PYGZsq{}}\PYG{p}{,} \PYG{l+s}{\PYGZsq{}}\PYG{l+s}{J}\PYG{l+s}{\PYGZsq{}}\PYG{p}{,} \PYG{l+s}{\PYGZsq{}}\PYG{l+s}{ja}\PYG{l+s}{\PYGZsq{}}\PYG{p}{,} \PYG{l+s}{\PYGZsq{}}\PYG{l+s}{Ja}\PYG{l+s}{\PYGZsq{}}\PYG{p}{)}\PYG{p}{:} \PYG{k}{return} \PYG{n+nb+bp}{True}
       \PYG{k}{if} \PYG{n}{ok} \PYG{o+ow}{in} \PYG{p}{(}\PYG{l+s}{\PYGZsq{}}\PYG{l+s}{n}\PYG{l+s}{\PYGZsq{}}\PYG{p}{,} \PYG{l+s}{\PYGZsq{}}\PYG{l+s}{N}\PYG{l+s}{\PYGZsq{}}\PYG{p}{,} \PYG{l+s}{\PYGZsq{}}\PYG{l+s}{ne}\PYG{l+s}{\PYGZsq{}}\PYG{p}{,} \PYG{l+s}{\PYGZsq{}}\PYG{l+s}{Ne}\PYG{l+s}{\PYGZsq{}}\PYG{p}{,} \PYG{l+s}{\PYGZsq{}}\PYG{l+s}{Nein}\PYG{l+s}{\PYGZsq{}}\PYG{p}{)}\PYG{p}{:} \PYG{k}{return} \PYG{n+nb+bp}{False}
       \PYG{n}{retries} \PYG{o}{=} \PYG{n}{retries} \PYG{o}{\PYGZhy{}} \PYG{l+m+mi}{1}
       \PYG{k}{if} \PYG{n}{retries} \PYG{o}{\PYGZlt{}} \PYG{l+m+mi}{0}\PYG{p}{:}
           \PYG{k}{raise} \PYG{n+ne}{IOError}\PYG{p}{(}\PYG{l+s}{\PYGZsq{}}\PYG{l+s}{Benutzer abgelehnt!}\PYG{l+s}{\PYGZsq{}}\PYG{p}{)}
       \PYG{k}{print}\PYG{p}{(}\PYG{n}{complaint}\PYG{p}{)}
\end{Verbatim}
\begin{itemize}
\item {} 
\code{ask\_ok("Datei überschreiben", retries=2)}

\item {} 
\code{ask\_ok("Datei überschreiben", complaint="Komm schon, nur Ja oder Nein")}

\end{itemize}


\chapter{Beliebig lange Argumentlisten}
\label{index:beliebig-lange-argumentlisten}
\begin{Verbatim}[commandchars=\\\{\}]
\PYG{g+gp}{\PYGZgt{}\PYGZgt{}\PYGZgt{} }\PYG{k}{def} \PYG{n+nf}{concat}\PYG{p}{(}\PYG{o}{*}\PYG{n}{args}\PYG{p}{,} \PYG{n}{sep}\PYG{o}{=}\PYG{l+s}{\PYGZdq{}}\PYG{l+s}{/}\PYG{l+s}{\PYGZdq{}}\PYG{p}{)}\PYG{p}{:}
\PYG{g+gp}{... }   \PYG{k}{return} \PYG{n}{sep}\PYG{o}{.}\PYG{n}{join}\PYG{p}{(}\PYG{n}{args}\PYG{p}{)}
\PYG{g+gp}{...}
\PYG{g+gp}{\PYGZgt{}\PYGZgt{}\PYGZgt{} }\PYG{n}{concat}\PYG{p}{(}\PYG{l+s}{\PYGZdq{}}\PYG{l+s}{Erde}\PYG{l+s}{\PYGZdq{}}\PYG{p}{,} \PYG{l+s}{\PYGZdq{}}\PYG{l+s}{Mars}\PYG{l+s}{\PYGZdq{}}\PYG{p}{,} \PYG{l+s}{\PYGZdq{}}\PYG{l+s}{Venus}\PYG{l+s}{\PYGZdq{}}\PYG{p}{)}
\PYG{g+go}{\PYGZsq{}Erde/Mars/Venus\PYGZsq{}}
\PYG{g+gp}{\PYGZgt{}\PYGZgt{}\PYGZgt{} }\PYG{n}{concat}\PYG{p}{(}\PYG{l+s}{\PYGZdq{}}\PYG{l+s}{Erde}\PYG{l+s}{\PYGZdq{}}\PYG{p}{,} \PYG{l+s}{\PYGZdq{}}\PYG{l+s}{Mars}\PYG{l+s}{\PYGZdq{}}\PYG{p}{,} \PYG{l+s}{\PYGZdq{}}\PYG{l+s}{Venus}\PYG{l+s}{\PYGZdq{}}\PYG{p}{,} \PYG{n}{sep}\PYG{o}{=}\PYG{l+s}{\PYGZdq{}}\PYG{l+s}{.}\PYG{l+s}{\PYGZdq{}}\PYG{p}{)}
\PYG{g+go}{\PYGZsq{}Erde.Mars.Venus\PYGZsq{}}
\end{Verbatim}


\chapter{Argumentlisten auspacken}
\label{index:argumentlisten-auspacken}
\begin{Verbatim}[commandchars=\\\{\}]
\PYG{g+gp}{\PYGZgt{}\PYGZgt{}\PYGZgt{} }\PYG{n+nb}{list}\PYG{p}{(}\PYG{n+nb}{range}\PYG{p}{(}\PYG{l+m+mi}{3}\PYG{p}{,} \PYG{l+m+mi}{6}\PYG{p}{)}\PYG{p}{)}   \PYG{c}{\PYGZsh{} normaler Aufruf mit getrennten Argumenten}
\PYG{g+go}{[3, 4, 5]}
\PYG{g+gp}{\PYGZgt{}\PYGZgt{}\PYGZgt{} }\PYG{n}{args} \PYG{o}{=} \PYG{p}{[}\PYG{l+m+mi}{3}\PYG{p}{,} \PYG{l+m+mi}{6}\PYG{p}{]}
\PYG{g+gp}{\PYGZgt{}\PYGZgt{}\PYGZgt{} }\PYG{n+nb}{list}\PYG{p}{(}\PYG{n+nb}{range}\PYG{p}{(}\PYG{o}{*}\PYG{n}{args}\PYG{p}{)}\PYG{p}{)}  \PYG{c}{\PYGZsh{} Aufruf mit Argumenten, die aus einer Liste ausgepackt werden}
\PYG{g+go}{[3, 4, 5]}
\end{Verbatim}


\chapter{Lambda-Form - anonyme Funktion}
\label{index:lambda-form-anonyme-funktion}
\begin{Verbatim}[commandchars=\\\{\}]
\PYG{g+gp}{\PYGZgt{}\PYGZgt{}\PYGZgt{} }\PYG{k}{def} \PYG{n+nf}{make\PYGZus{}incrementor}\PYG{p}{(}\PYG{n}{n}\PYG{p}{)}\PYG{p}{:}
\PYG{g+gp}{... }    \PYG{k}{return} \PYG{k}{lambda} \PYG{n}{x}\PYG{p}{:} \PYG{n}{x} \PYG{o}{+} \PYG{n}{n}
\PYG{g+gp}{...}
\PYG{g+gp}{\PYGZgt{}\PYGZgt{}\PYGZgt{} }\PYG{n}{add42} \PYG{o}{=} \PYG{n}{make\PYGZus{}incrementor}\PYG{p}{(}\PYG{l+m+mi}{42}\PYG{p}{)}
\PYG{g+gp}{\PYGZgt{}\PYGZgt{}\PYGZgt{} }\PYG{n}{add42}\PYG{p}{(}\PYG{l+m+mi}{0}\PYG{p}{)}
\PYG{g+go}{42}
\PYG{g+gp}{\PYGZgt{}\PYGZgt{}\PYGZgt{} }\PYG{n}{add42}\PYG{p}{(}\PYG{l+m+mi}{1}\PYG{p}{)}
\PYG{g+go}{43}
\end{Verbatim}


\chapter{Dokumentationsstrings}
\label{index:dokumentationsstrings}
\begin{Verbatim}[commandchars=\\\{\}]
\PYG{g+gp}{\PYGZgt{}\PYGZgt{}\PYGZgt{} }\PYG{k}{def} \PYG{n+nf}{my\PYGZus{}function}\PYG{p}{(}\PYG{p}{)}\PYG{p}{:}
\PYG{g+gp}{... }    \PYG{l+s+sd}{\PYGZdq{}\PYGZdq{}\PYGZdq{}Do nothing, but document it.}
\PYG{g+gp}{... }\PYG{l+s+sd}{    No, really, it doesn\PYGZsq{}t do anything.}
\PYG{g+gp}{... }\PYG{l+s+sd}{    \PYGZdq{}\PYGZdq{}\PYGZdq{}}
\PYG{g+gp}{... }    \PYG{k}{pass}
\PYG{g+gp}{...}
\PYG{g+gp}{\PYGZgt{}\PYGZgt{}\PYGZgt{} }\PYG{k}{print}\PYG{p}{(}\PYG{n}{my\PYGZus{}function}\PYG{o}{.}\PYG{n}{\PYGZus{}\PYGZus{}doc\PYGZus{}\PYGZus{}}\PYG{p}{)}
\PYG{g+go}{Do nothing, but document it.}
\PYG{g+go}{   No, really, it doesn\PYGZsq{}t do anything.}
\end{Verbatim}


\chapter{\index{Python Enhancement Proposals!PEP 8}\textbf{PEP 8}}
\label{index:pep-8}\begin{itemize}
\item {} 
4 Leerzeichen, keine Tabs

\item {} 
maximale Zeilenlänge: 79 Zeichen

\item {} 
Leerzeilen

\item {} 
eigene Zeile für Kommentare, sofern möglich

\item {} 
Docstrings

\item {} 
Leerzeichen um Operatoren herum und nach Kommas, jedoch nicht direkt
innerhalb von Klammerkonstrukten: \code{a = f(1, 2) + g(3, 4)}

\item {} 
\code{CamelCase} für Klassen

\item {} 
\code{klein\_geschrieben\_mit\_unterstrichen} für Funktionen und Methoden

\item {} 
UTF-8 -- oder sogar einfaches ASCII

\item {} 
möglichst keine nicht-ASCII-Zeichen in Bezeichnern

\end{itemize}


\chapter{Datenstrukturen}
\label{index:datenstrukturen}

\chapter{Mehr zu Listen}
\label{index:mehr-zu-listen}
\begin{Verbatim}[commandchars=\\\{\}]
\PYG{g+gp}{\PYGZgt{}\PYGZgt{}\PYGZgt{} }\PYG{n}{a} \PYG{o}{=} \PYG{p}{[}\PYG{l+m+mf}{66.25}\PYG{p}{,} \PYG{l+m+mi}{333}\PYG{p}{,} \PYG{l+m+mi}{333}\PYG{p}{,} \PYG{l+m+mi}{1}\PYG{p}{,} \PYG{l+m+mf}{1234.5}\PYG{p}{]}
\PYG{g+gp}{\PYGZgt{}\PYGZgt{}\PYGZgt{} }\PYG{k}{print}\PYG{p}{(}\PYG{n}{a}\PYG{o}{.}\PYG{n}{count}\PYG{p}{(}\PYG{l+m+mi}{333}\PYG{p}{)}\PYG{p}{,} \PYG{n}{a}\PYG{o}{.}\PYG{n}{count}\PYG{p}{(}\PYG{l+m+mf}{66.25}\PYG{p}{)}\PYG{p}{,} \PYG{n}{a}\PYG{o}{.}\PYG{n}{count}\PYG{p}{(}\PYG{l+s}{\PYGZsq{}}\PYG{l+s}{x}\PYG{l+s}{\PYGZsq{}}\PYG{p}{)}\PYG{p}{)}
\PYG{g+go}{2 1 0}
\end{Verbatim}

\begin{Verbatim}[commandchars=\\\{\}]
\PYG{g+gp}{\PYGZgt{}\PYGZgt{}\PYGZgt{} }\PYG{n}{a}\PYG{o}{.}\PYG{n}{insert}\PYG{p}{(}\PYG{l+m+mi}{2}\PYG{p}{,} \PYG{o}{\PYGZhy{}}\PYG{l+m+mi}{1}\PYG{p}{)}
\PYG{g+gp}{\PYGZgt{}\PYGZgt{}\PYGZgt{} }\PYG{n}{a}\PYG{o}{.}\PYG{n}{append}\PYG{p}{(}\PYG{l+m+mi}{333}\PYG{p}{)}
\PYG{g+gp}{\PYGZgt{}\PYGZgt{}\PYGZgt{} }\PYG{n}{a}
\PYG{g+go}{[66.25, 333, \PYGZhy{}1, 333, 1, 1234.5, 333]}
\end{Verbatim}

\begin{Verbatim}[commandchars=\\\{\}]
\PYG{g+gp}{\PYGZgt{}\PYGZgt{}\PYGZgt{} }\PYG{n}{a}\PYG{o}{.}\PYG{n}{index}\PYG{p}{(}\PYG{l+m+mi}{333}\PYG{p}{)}
\PYG{g+go}{1}
\end{Verbatim}

\begin{Verbatim}[commandchars=\\\{\}]
\PYG{g+gp}{\PYGZgt{}\PYGZgt{}\PYGZgt{} }\PYG{n}{a}\PYG{o}{.}\PYG{n}{remove}\PYG{p}{(}\PYG{l+m+mi}{333}\PYG{p}{)}
\PYG{g+gp}{\PYGZgt{}\PYGZgt{}\PYGZgt{} }\PYG{n}{a}
\PYG{g+go}{[66.25, \PYGZhy{}1, 333, 1, 1234.5, 333]}
\end{Verbatim}


\chapter{Mehr zu Listen}
\label{index:id5}
\begin{Verbatim}[commandchars=\\\{\}]
\PYG{g+gp}{\PYGZgt{}\PYGZgt{}\PYGZgt{} }\PYG{n}{a}\PYG{o}{.}\PYG{n}{reverse}\PYG{p}{(}\PYG{p}{)}
\PYG{g+gp}{\PYGZgt{}\PYGZgt{}\PYGZgt{} }\PYG{n}{a}
\PYG{g+go}{[333, 1234.5, 1, 333, \PYGZhy{}1, 66.25]}
\end{Verbatim}

\begin{Verbatim}[commandchars=\\\{\}]
\PYG{g+gp}{\PYGZgt{}\PYGZgt{}\PYGZgt{} }\PYG{n}{a}\PYG{o}{.}\PYG{n}{sort}\PYG{p}{(}\PYG{p}{)}
\PYG{g+gp}{\PYGZgt{}\PYGZgt{}\PYGZgt{} }\PYG{n}{a}
\PYG{g+go}{[\PYGZhy{}1, 1, 66.25, 333, 333, 1234.5]}
\end{Verbatim}


\chapter{List Comprehensions}
\label{index:list-comprehensions}
Listen aus Listen erzeugen

\begin{Verbatim}[commandchars=\\\{\}]
\PYG{g+gp}{\PYGZgt{}\PYGZgt{}\PYGZgt{} }\PYG{n}{vec} \PYG{o}{=} \PYG{p}{[}\PYG{l+m+mi}{2}\PYG{p}{,} \PYG{l+m+mi}{4}\PYG{p}{,} \PYG{l+m+mi}{7}\PYG{p}{]}
\PYG{g+gp}{\PYGZgt{}\PYGZgt{}\PYGZgt{} }\PYG{n}{vec3} \PYG{o}{=} \PYG{p}{[}\PYG{p}{]}
\PYG{g+gp}{\PYGZgt{}\PYGZgt{}\PYGZgt{} }\PYG{k}{for} \PYG{n}{x} \PYG{o+ow}{in} \PYG{n}{vec}\PYG{p}{:}
\PYG{g+gp}{... }    \PYG{n}{vec3}\PYG{o}{.}\PYG{n}{append}\PYG{p}{(}\PYG{l+m+mi}{3}\PYG{o}{*}\PYG{n}{x}\PYG{p}{)}
\PYG{g+gp}{\PYGZgt{}\PYGZgt{}\PYGZgt{} }\PYG{n}{vec3}
\PYG{g+go}{[6, 12, 21]}
\end{Verbatim}


\chapter{List Comprehensions}
\label{index:id6}
\begin{Verbatim}[commandchars=\\\{\}]
\PYG{g+gp}{\PYGZgt{}\PYGZgt{}\PYGZgt{} }\PYG{n}{vec} \PYG{o}{=} \PYG{p}{[}\PYG{l+m+mi}{2}\PYG{p}{,} \PYG{l+m+mi}{4}\PYG{p}{,} \PYG{l+m+mi}{7}\PYG{p}{]}
\PYG{g+gp}{\PYGZgt{}\PYGZgt{}\PYGZgt{} }\PYG{p}{[}\PYG{l+m+mi}{3}\PYG{o}{*}\PYG{n}{x} \PYG{k}{for} \PYG{n}{x} \PYG{o+ow}{in} \PYG{n}{vec}\PYG{p}{]}
\PYG{g+go}{[6, 12, 21]}
\end{Verbatim}

Jetzt wird's ein wenig ausgefallener:

\begin{Verbatim}[commandchars=\\\{\}]
\PYG{g+gp}{\PYGZgt{}\PYGZgt{}\PYGZgt{} }\PYG{p}{[}\PYG{p}{[}\PYG{n}{x}\PYG{p}{,} \PYG{n}{x}\PYG{o}{*}\PYG{o}{*}\PYG{l+m+mi}{2}\PYG{p}{]} \PYG{k}{for} \PYG{n}{x} \PYG{o+ow}{in} \PYG{n}{vec}\PYG{p}{]}
\PYG{g+go}{[[2, 4], [4, 16], [7, 49]]}
\end{Verbatim}


\chapter{List Comprehensions (2)}
\label{index:list-comprehensions-2}
Hier wenden wir einen Methodenaufruf auf jedes Objekt in der Sequenz an:

\begin{Verbatim}[commandchars=\\\{\}]
\PYG{g+gp}{\PYGZgt{}\PYGZgt{}\PYGZgt{} }\PYG{n}{freshfruits} \PYG{o}{=} \PYG{p}{[}\PYG{l+s}{\PYGZsq{}}\PYG{l+s}{  banana}\PYG{l+s}{\PYGZsq{}}\PYG{p}{,} \PYG{l+s}{\PYGZsq{}}\PYG{l+s}{  loganberry }\PYG{l+s}{\PYGZsq{}}\PYG{p}{,} \PYG{l+s}{\PYGZsq{}}\PYG{l+s}{passion fruit  }\PYG{l+s}{\PYGZsq{}}\PYG{p}{]}
\PYG{g+gp}{\PYGZgt{}\PYGZgt{}\PYGZgt{} }\PYG{p}{[}\PYG{n}{fruit}\PYG{o}{.}\PYG{n}{strip}\PYG{p}{(}\PYG{p}{)} \PYG{k}{for} \PYG{n}{fruit} \PYG{o+ow}{in} \PYG{n}{freshfruits}\PYG{p}{]}
\PYG{g+go}{[\PYGZsq{}banana\PYGZsq{}, \PYGZsq{}loganberry\PYGZsq{}, \PYGZsq{}passion fruit\PYGZsq{}]}
\end{Verbatim}


\chapter{List Comprehensions mit if}
\label{index:list-comprehensions-mit-if}
Indem wir eine \href{https://docs.python.org/reference/compound\_stmts.html\#if}{\code{if}}-Klausel anwenden, können wir die Elemente filtern:

\begin{Verbatim}[commandchars=\\\{\}]
\PYG{g+gp}{\PYGZgt{}\PYGZgt{}\PYGZgt{} }\PYG{p}{[}\PYG{l+m+mi}{3}\PYG{o}{*}\PYG{n}{x} \PYG{k}{for} \PYG{n}{x} \PYG{o+ow}{in} \PYG{n}{vec} \PYG{k}{if} \PYG{n}{x} \PYG{o}{\PYGZgt{}} \PYG{l+m+mi}{3}\PYG{p}{]}
\PYG{g+go}{[12, 18]}
\PYG{g+gp}{\PYGZgt{}\PYGZgt{}\PYGZgt{} }\PYG{p}{[}\PYG{l+m+mi}{3}\PYG{o}{*}\PYG{n}{x} \PYG{k}{for} \PYG{n}{x} \PYG{o+ow}{in} \PYG{n}{vec} \PYG{k}{if} \PYG{n}{x} \PYG{o}{\PYGZlt{}} \PYG{l+m+mi}{2}\PYG{p}{]}
\PYG{g+go}{[]}
\end{Verbatim}


\chapter{Verschachtelte List Comprehensions}
\label{index:verschachtelte-list-comprehensions}
Lese von rechts nach links

\begin{Verbatim}[commandchars=\\\{\}]
\PYG{g+gp}{\PYGZgt{}\PYGZgt{}\PYGZgt{} }\PYG{n}{mat} \PYG{o}{=} \PYG{p}{[}
\PYG{g+gp}{... }       \PYG{p}{[}\PYG{l+m+mi}{1}\PYG{p}{,} \PYG{l+m+mi}{2}\PYG{p}{,} \PYG{l+m+mi}{3}\PYG{p}{]}\PYG{p}{,}
\PYG{g+gp}{... }       \PYG{p}{[}\PYG{l+m+mi}{4}\PYG{p}{,} \PYG{l+m+mi}{5}\PYG{p}{,} \PYG{l+m+mi}{6}\PYG{p}{]}\PYG{p}{,}
\PYG{g+gp}{... }       \PYG{p}{[}\PYG{l+m+mi}{7}\PYG{p}{,} \PYG{l+m+mi}{8}\PYG{p}{,} \PYG{l+m+mi}{9}\PYG{p}{]}\PYG{p}{,}
\PYG{g+gp}{... }      \PYG{p}{]}
\end{Verbatim}

\begin{Verbatim}[commandchars=\\\{\}]
\PYG{g+gp}{\PYGZgt{}\PYGZgt{}\PYGZgt{} }\PYG{k}{print}\PYG{p}{(}\PYG{p}{[}\PYG{p}{[}\PYG{n}{row}\PYG{p}{[}\PYG{n}{i}\PYG{p}{]} \PYG{k}{for} \PYG{n}{row} \PYG{o+ow}{in} \PYG{n}{mat}\PYG{p}{]} \PYG{k}{for} \PYG{n}{i} \PYG{o+ow}{in} \PYG{p}{[}\PYG{l+m+mi}{0}\PYG{p}{,} \PYG{l+m+mi}{1}\PYG{p}{,} \PYG{l+m+mi}{2}\PYG{p}{]}\PYG{p}{]}\PYG{p}{)}
\PYG{g+go}{[[1, 4, 7], [2, 5, 8], [3, 6, 9]]}
\end{Verbatim}

\begin{notice}{note}{Bemerkung:}
Beispiel: 3x3 Matrix invertieren

Eine ausführlichere Version dieses Schnipsels zeigt den Ablauf deutlich:

\begin{Verbatim}[commandchars=\\\{\}]
for i in [0, 1, 2]:
    for row in mat:
        print(row[i], end=\PYGZdq{}\PYGZdq{})
    print()
\end{Verbatim}

Im echten Leben sollte man aber eingebaute Funktionen komplexen Anweisungen
vorziehen. Die Funktion \href{https://docs.python.org/library/functions.html\#zip}{\code{zip()}} würde in diesem Fall gute Dienste leisten

\begin{Verbatim}[commandchars=\\\{\}]
\PYG{g+gp}{\PYGZgt{}\PYGZgt{}\PYGZgt{} }\PYG{n+nb}{list}\PYG{p}{(}\PYG{n+nb}{zip}\PYG{p}{(}\PYG{o}{*}\PYG{n}{mat}\PYG{p}{)}\PYG{p}{)}
\PYG{g+go}{[(1, 4, 7), (2, 5, 8), (3, 6, 9)]}
\end{Verbatim}
\end{notice}


\chapter{List comprehensions}
\label{index:id7}
Hier sind ein paar verschachtelte \href{https://docs.python.org/reference/compound\_stmts.html\#for}{\code{for}}-Schleifen und anderes
ausgefallenes Verhalten:

\begin{Verbatim}[commandchars=\\\{\}]
\PYG{g+gp}{\PYGZgt{}\PYGZgt{}\PYGZgt{} }\PYG{n}{vec1} \PYG{o}{=} \PYG{p}{[}\PYG{l+m+mi}{2}\PYG{p}{,} \PYG{l+m+mi}{4}\PYG{p}{,} \PYG{l+m+mi}{6}\PYG{p}{]}
\PYG{g+gp}{\PYGZgt{}\PYGZgt{}\PYGZgt{} }\PYG{n}{vec2} \PYG{o}{=} \PYG{p}{[}\PYG{l+m+mi}{4}\PYG{p}{,} \PYG{l+m+mi}{3}\PYG{p}{,} \PYG{o}{\PYGZhy{}}\PYG{l+m+mi}{9}\PYG{p}{]}
\PYG{g+gp}{\PYGZgt{}\PYGZgt{}\PYGZgt{} }\PYG{p}{[}\PYG{n}{x}\PYG{o}{*}\PYG{n}{y} \PYG{k}{for} \PYG{n}{x} \PYG{o+ow}{in} \PYG{n}{vec1} \PYG{k}{for} \PYG{n}{y} \PYG{o+ow}{in} \PYG{n}{vec2}\PYG{p}{]}
\PYG{g+go}{[8, 6, \PYGZhy{}18, 16, 12, \PYGZhy{}36, 24, 18, \PYGZhy{}54]}
\PYG{g+gp}{\PYGZgt{}\PYGZgt{}\PYGZgt{} }\PYG{p}{[}\PYG{n}{x}\PYG{o}{+}\PYG{n}{y} \PYG{k}{for} \PYG{n}{x} \PYG{o+ow}{in} \PYG{n}{vec1} \PYG{k}{for} \PYG{n}{y} \PYG{o+ow}{in} \PYG{n}{vec2}\PYG{p}{]}
\PYG{g+go}{[6, 5, \PYGZhy{}7, 8, 7, \PYGZhy{}5, 10, 9, \PYGZhy{}3]}
\PYG{g+gp}{\PYGZgt{}\PYGZgt{}\PYGZgt{} }\PYG{p}{[}\PYG{n}{vec1}\PYG{p}{[}\PYG{n}{i}\PYG{p}{]}\PYG{o}{*}\PYG{n}{vec2}\PYG{p}{[}\PYG{n}{i}\PYG{p}{]} \PYG{k}{for} \PYG{n}{i} \PYG{o+ow}{in} \PYG{n+nb}{range}\PYG{p}{(}\PYG{n+nb}{len}\PYG{p}{(}\PYG{n}{vec1}\PYG{p}{)}\PYG{p}{)}\PYG{p}{]}
\PYG{g+go}{[8, 12, \PYGZhy{}54]}
\end{Verbatim}

List Comprehensions können auf komplexe Ausdrücke und verschachtelte Funktionen
angewendet werden:

\begin{Verbatim}[commandchars=\\\{\}]
\PYG{g+gp}{\PYGZgt{}\PYGZgt{}\PYGZgt{} }\PYG{p}{[}\PYG{n+nb}{str}\PYG{p}{(}\PYG{n+nb}{round}\PYG{p}{(}\PYG{l+m+mi}{355}\PYG{o}{/}\PYG{l+m+mi}{113}\PYG{p}{,} \PYG{n}{i}\PYG{p}{)}\PYG{p}{)} \PYG{k}{for} \PYG{n}{i} \PYG{o+ow}{in} \PYG{n+nb}{range}\PYG{p}{(}\PYG{l+m+mi}{1}\PYG{p}{,} \PYG{l+m+mi}{6}\PYG{p}{)}\PYG{p}{]}
\PYG{g+go}{[\PYGZsq{}3.1\PYGZsq{}, \PYGZsq{}3.14\PYGZsq{}, \PYGZsq{}3.142\PYGZsq{}, \PYGZsq{}3.1416\PYGZsq{}, \PYGZsq{}3.14159\PYGZsq{}]}
\end{Verbatim}


\chapter{Dict Comprehensions}
\label{index:dict-comprehensions}
\begin{Verbatim}[commandchars=\\\{\}]
\PYG{g+gp}{\PYGZgt{}\PYGZgt{}\PYGZgt{} }\PYG{p}{\PYGZob{}}\PYG{n}{x}\PYG{p}{:} \PYG{n}{x}\PYG{o}{*}\PYG{o}{*}\PYG{l+m+mi}{2} \PYG{k}{for} \PYG{n}{x} \PYG{o+ow}{in} \PYG{p}{(}\PYG{l+m+mi}{2}\PYG{p}{,} \PYG{l+m+mi}{4}\PYG{p}{,} \PYG{l+m+mi}{6}\PYG{p}{)}\PYG{p}{\PYGZcb{}}
\PYG{g+go}{\PYGZob{}2: 4, 4: 16, 6: 36\PYGZcb{}}
\PYG{g+gp}{\PYGZgt{}\PYGZgt{}\PYGZgt{} }\PYG{p}{\PYGZob{}}\PYG{n}{w}\PYG{p}{:} \PYG{n+nb}{len}\PYG{p}{(}\PYG{n}{w}\PYG{p}{)} \PYG{k}{for} \PYG{n}{w} \PYG{o+ow}{in} \PYG{p}{[}\PYG{l+s}{\PYGZsq{}}\PYG{l+s}{Linux}\PYG{l+s}{\PYGZsq{}}\PYG{p}{,} \PYG{l+s}{\PYGZsq{}}\PYG{l+s}{Windows}\PYG{l+s}{\PYGZsq{}}\PYG{p}{,} \PYG{l+s}{\PYGZsq{}}\PYG{l+s}{Mac OS X}\PYG{l+s}{\PYGZsq{}}\PYG{p}{]}\PYG{p}{\PYGZcb{}}
\PYG{g+go}{\PYGZob{}\PYGZsq{}Windows\PYGZsq{}: 7, \PYGZsq{}Mac OS X\PYGZsq{}: 8, \PYGZsq{}Linux\PYGZsq{}: 5\PYGZcb{}}
\end{Verbatim}


\chapter{Die \texttt{del}-Anweisung}
\label{index:die-del-anweisung}
Löschen von Listenelement nach index:

\begin{Verbatim}[commandchars=\\\{\}]
\PYG{g+gp}{\PYGZgt{}\PYGZgt{}\PYGZgt{} }\PYG{n}{a} \PYG{o}{=} \PYG{p}{[}\PYG{o}{\PYGZhy{}}\PYG{l+m+mi}{1}\PYG{p}{,} \PYG{l+m+mi}{1}\PYG{p}{,} \PYG{l+m+mf}{66.25}\PYG{p}{,} \PYG{l+m+mi}{333}\PYG{p}{,} \PYG{l+m+mi}{333}\PYG{p}{,} \PYG{l+m+mf}{1234.5}\PYG{p}{]}
\PYG{g+gp}{\PYGZgt{}\PYGZgt{}\PYGZgt{} }\PYG{k}{del} \PYG{n}{a}\PYG{p}{[}\PYG{l+m+mi}{0}\PYG{p}{]}
\PYG{g+gp}{\PYGZgt{}\PYGZgt{}\PYGZgt{} }\PYG{n}{a}
\PYG{g+go}{[1, 66.25, 333, 333, 1234.5]}
\PYG{g+gp}{\PYGZgt{}\PYGZgt{}\PYGZgt{} }\PYG{k}{del} \PYG{n}{a}\PYG{p}{[}\PYG{l+m+mi}{2}\PYG{p}{:}\PYG{l+m+mi}{4}\PYG{p}{]}
\PYG{g+gp}{\PYGZgt{}\PYGZgt{}\PYGZgt{} }\PYG{n}{a}
\PYG{g+go}{[1, 66.25, 1234.5]}
\PYG{g+gp}{\PYGZgt{}\PYGZgt{}\PYGZgt{} }\PYG{k}{del} \PYG{n}{a}\PYG{p}{[}\PYG{p}{:}\PYG{p}{]}
\PYG{g+gp}{\PYGZgt{}\PYGZgt{}\PYGZgt{} }\PYG{n}{a}
\PYG{g+go}{[]}
\end{Verbatim}


\chapter{Die \texttt{del}-Anweisung (2)}
\label{index:die-del-anweisung-2}
Löschen von ganzen Variablen:

\begin{Verbatim}[commandchars=\\\{\}]
\PYG{g+gp}{\PYGZgt{}\PYGZgt{}\PYGZgt{} }\PYG{k}{del} \PYG{n}{a}
\PYG{g+gp}{\PYGZgt{}\PYGZgt{}\PYGZgt{} }\PYG{n}{a}
\PYG{g+gt}{Traceback (most recent call last):}
  File \PYG{n+nb}{\PYGZdq{}\PYGZlt{}stdin\PYGZgt{}\PYGZdq{}}, line \PYG{l+m}{1}, in \PYG{n}{\PYGZlt{}module\PYGZgt{}}
\PYG{g+gr}{NameError}: \PYG{n}{name \PYGZsq{}a\PYGZsq{} is not defined}
\end{Verbatim}


\chapter{Tupel}
\label{index:tupel}
Ein Tupel besteht aus mehreren Werten, die durch Kommas von einander getrennt
sind, beispielsweise:

\begin{Verbatim}[commandchars=\\\{\}]
\PYG{g+gp}{\PYGZgt{}\PYGZgt{}\PYGZgt{} }\PYG{n}{t} \PYG{o}{=} \PYG{l+m+mi}{12345}\PYG{p}{,} \PYG{l+m+mi}{54321}\PYG{p}{,} \PYG{l+s}{\PYGZsq{}}\PYG{l+s}{Hallo!}\PYG{l+s}{\PYGZsq{}}
\PYG{g+gp}{\PYGZgt{}\PYGZgt{}\PYGZgt{} }\PYG{n}{t}\PYG{p}{[}\PYG{l+m+mi}{0}\PYG{p}{]}
\PYG{g+go}{12345}
\PYG{g+gp}{\PYGZgt{}\PYGZgt{}\PYGZgt{} }\PYG{n}{t}
\PYG{g+go}{(12345, 54321, \PYGZsq{}Hallo!\PYGZsq{})}
\PYG{g+gp}{\PYGZgt{}\PYGZgt{}\PYGZgt{} }\PYG{c}{\PYGZsh{} Tupel können verschachtelt werden:}
\PYG{g+gp}{... }\PYG{n}{u} \PYG{o}{=} \PYG{n}{t}\PYG{p}{,} \PYG{p}{(}\PYG{l+m+mi}{1}\PYG{p}{,} \PYG{l+m+mi}{2}\PYG{p}{,} \PYG{l+m+mi}{3}\PYG{p}{,} \PYG{l+m+mi}{4}\PYG{p}{,} \PYG{l+m+mi}{5}\PYG{p}{)}
\PYG{g+gp}{\PYGZgt{}\PYGZgt{}\PYGZgt{} }\PYG{n}{u}
\PYG{g+go}{((12345, 54321, \PYGZsq{}Hallo!\PYGZsq{}), (1, 2, 3, 4, 5))}
\end{Verbatim}

\begin{notice}{note}{Bemerkung:}\begin{itemize}
\item {} 
Listen und Zeichenketten sind Sequenztypen, Tupel auch

\end{itemize}
\end{notice}


\chapter{Tupel (2)}
\label{index:tupel-2}
Vorsicht bei Tupeln der Länge 0 und 1

\begin{Verbatim}[commandchars=\\\{\}]
\PYG{g+gp}{\PYGZgt{}\PYGZgt{}\PYGZgt{} }\PYG{n}{empty} \PYG{o}{=} \PYG{p}{(}\PYG{p}{)}
\PYG{g+gp}{\PYGZgt{}\PYGZgt{}\PYGZgt{} }\PYG{n}{singleton} \PYG{o}{=} \PYG{l+s}{\PYGZsq{}}\PYG{l+s}{Hallo}\PYG{l+s}{\PYGZsq{}}\PYG{p}{,}    \PYG{c}{\PYGZsh{} \PYGZlt{}\PYGZhy{}\PYGZhy{} das angehängte Komma nicht vergessen}
\PYG{g+gp}{\PYGZgt{}\PYGZgt{}\PYGZgt{} }\PYG{n}{x} \PYG{o}{=} \PYG{p}{(}\PYG{l+m+mi}{0}\PYG{p}{)}
\PYG{g+gp}{\PYGZgt{}\PYGZgt{}\PYGZgt{} }\PYG{n+nb}{len}\PYG{p}{(}\PYG{n}{empty}\PYG{p}{)}
\PYG{g+go}{0}
\PYG{g+gp}{\PYGZgt{}\PYGZgt{}\PYGZgt{} }\PYG{n+nb}{len}\PYG{p}{(}\PYG{n}{singleton}\PYG{p}{)}
\PYG{g+go}{1}
\PYG{g+gp}{\PYGZgt{}\PYGZgt{}\PYGZgt{} }\PYG{n}{singleton}
\PYG{g+go}{(\PYGZsq{}Hallo\PYGZsq{},)}
\PYG{g+gp}{\PYGZgt{}\PYGZgt{}\PYGZgt{} }\PYG{n+nb}{len}\PYG{p}{(}\PYG{n}{x}\PYG{p}{)}
\PYG{g+gt}{Traceback (most recent call last):}
  File \PYG{n+nb}{\PYGZdq{}\PYGZlt{}stdin\PYGZgt{}\PYGZdq{}}, line \PYG{l+m}{1}, in \PYG{n}{\PYGZlt{}module\PYGZgt{}}
\PYG{g+gr}{TypeError}: \PYG{n}{object of type \PYGZsq{}int\PYGZsq{} has no len()}
\end{Verbatim}


\chapter{Tupel packen und entpacken}
\label{index:tupel-packen-und-entpacken}
\begin{Verbatim}[commandchars=\\\{\}]
\PYG{g+gp}{\PYGZgt{}\PYGZgt{}\PYGZgt{} }\PYG{n}{t} \PYG{o}{=} \PYG{l+m+mi}{12345}\PYG{p}{,} \PYG{l+m+mi}{54321}\PYG{p}{,} \PYG{l+s}{\PYGZsq{}}\PYG{l+s}{Hallo!}\PYG{l+s}{\PYGZsq{}}
\PYG{g+gp}{\PYGZgt{}\PYGZgt{}\PYGZgt{} }\PYG{n}{x}\PYG{p}{,} \PYG{n}{y}\PYG{p}{,} \PYG{n}{z} \PYG{o}{=} \PYG{n}{t}
\end{Verbatim}


\chapter{Mengen (Sets)}
\label{index:mengen-sets}\begin{itemize}
\item {} 
ungeordnete Sammlung ohne doppelte Elemente

\item {} 
unterstützen mathematische Operationen
\begin{itemize}
\item {} 
Vereinigungsmenge

\item {} 
Schnittmenge

\item {} 
Differenz

\item {} 
symmetrische Differenz.

\end{itemize}

\end{itemize}


\chapter{Mengen (Sets)}
\label{index:id8}
\begin{Verbatim}[commandchars=\\\{\}]
\PYG{g+gp}{\PYGZgt{}\PYGZgt{}\PYGZgt{} }\PYG{n}{basket} \PYG{o}{=} \PYG{p}{\PYGZob{}}\PYG{l+s}{\PYGZsq{}}\PYG{l+s}{Apfel}\PYG{l+s}{\PYGZsq{}}\PYG{p}{,} \PYG{l+s}{\PYGZsq{}}\PYG{l+s}{Orange}\PYG{l+s}{\PYGZsq{}}\PYG{p}{,} \PYG{l+s}{\PYGZsq{}}\PYG{l+s}{Apfel}\PYG{l+s}{\PYGZsq{}}\PYG{p}{,} \PYG{l+s}{\PYGZsq{}}\PYG{l+s}{Birne}\PYG{l+s}{\PYGZsq{}}\PYG{p}{,} \PYG{l+s}{\PYGZsq{}}\PYG{l+s}{Orange}\PYG{l+s}{\PYGZsq{}}\PYG{p}{,} \PYG{l+s}{\PYGZsq{}}\PYG{l+s}{Banane}\PYG{l+s}{\PYGZsq{}}\PYG{p}{\PYGZcb{}}
\PYG{g+gp}{\PYGZgt{}\PYGZgt{}\PYGZgt{} }\PYG{k}{print}\PYG{p}{(}\PYG{n}{fruit}\PYG{p}{)}                       \PYG{c}{\PYGZsh{} zeigt, dass die Duplikate entfernt wurden}
\PYG{g+go}{\PYGZob{}\PYGZsq{}Orange\PYGZsq{}, \PYGZsq{}Birne\PYGZsq{}, \PYGZsq{}Apfel\PYGZsq{}, \PYGZsq{}Banane\PYGZsq{}\PYGZcb{}}
\PYG{g+gp}{\PYGZgt{}\PYGZgt{}\PYGZgt{} }\PYG{l+s}{\PYGZsq{}}\PYG{l+s}{Orange}\PYG{l+s}{\PYGZsq{}} \PYG{o+ow}{in} \PYG{n}{basket}                 \PYG{c}{\PYGZsh{} schnelles Testen auf Mitgliedschaft}
\PYG{g+go}{True}
\PYG{g+gp}{\PYGZgt{}\PYGZgt{}\PYGZgt{} }\PYG{l+s}{\PYGZsq{}}\PYG{l+s}{Fingerhirse}\PYG{l+s}{\PYGZsq{}} \PYG{o+ow}{in} \PYG{n}{basket}
\PYG{g+go}{False}
\end{Verbatim}


\chapter{Mengen (Sets)}
\label{index:id9}
\begin{Verbatim}[commandchars=\\\{\}]
\PYG{g+gp}{\PYGZgt{}\PYGZgt{}\PYGZgt{} }\PYG{n}{a} \PYG{o}{=} \PYG{n+nb}{set}\PYG{p}{(}\PYG{l+s}{\PYGZsq{}}\PYG{l+s}{abracadabra}\PYG{l+s}{\PYGZsq{}}\PYG{p}{)}
\PYG{g+gp}{\PYGZgt{}\PYGZgt{}\PYGZgt{} }\PYG{n}{b} \PYG{o}{=} \PYG{n+nb}{set}\PYG{p}{(}\PYG{l+s}{\PYGZsq{}}\PYG{l+s}{alacazam}\PYG{l+s}{\PYGZsq{}}\PYG{p}{)}
\PYG{g+gp}{\PYGZgt{}\PYGZgt{}\PYGZgt{} }\PYG{n}{a}                                  \PYG{c}{\PYGZsh{} einzelne Buchstaben in a}
\PYG{g+go}{\PYGZob{}\PYGZsq{}a\PYGZsq{}, \PYGZsq{}r\PYGZsq{}, \PYGZsq{}b\PYGZsq{}, \PYGZsq{}c\PYGZsq{}, \PYGZsq{}d\PYGZsq{}\PYGZcb{}}
\PYG{g+gp}{\PYGZgt{}\PYGZgt{}\PYGZgt{} }\PYG{n}{a} \PYG{o}{\PYGZhy{}} \PYG{n}{b}                              \PYG{c}{\PYGZsh{} in a aber nicht in b}
\PYG{g+go}{\PYGZob{}\PYGZsq{}r\PYGZsq{}, \PYGZsq{}d\PYGZsq{}, \PYGZsq{}b\PYGZsq{}\PYGZcb{}}
\PYG{g+gp}{\PYGZgt{}\PYGZgt{}\PYGZgt{} }\PYG{n}{a} \PYG{o}{\textbar{}} \PYG{n}{b}                              \PYG{c}{\PYGZsh{} in a oder b}
\PYG{g+go}{\PYGZob{}\PYGZsq{}a\PYGZsq{}, \PYGZsq{}c\PYGZsq{}, \PYGZsq{}r\PYGZsq{}, \PYGZsq{}d\PYGZsq{}, \PYGZsq{}b\PYGZsq{}, \PYGZsq{}m\PYGZsq{}, \PYGZsq{}z\PYGZsq{}, \PYGZsq{}l\PYGZsq{}\PYGZcb{}}
\PYG{g+gp}{\PYGZgt{}\PYGZgt{}\PYGZgt{} }\PYG{n}{a} \PYG{o}{\PYGZam{}} \PYG{n}{b}                              \PYG{c}{\PYGZsh{} sowohl in a, als auch in b}
\PYG{g+go}{\PYGZob{}\PYGZsq{}a\PYGZsq{}, \PYGZsq{}c\PYGZsq{}\PYGZcb{}}
\PYG{g+gp}{\PYGZgt{}\PYGZgt{}\PYGZgt{} }\PYG{n}{a} \PYG{o}{\PYGZca{}} \PYG{n}{b}                              \PYG{c}{\PYGZsh{} entweder in a oder b}
\PYG{g+go}{\PYGZob{}\PYGZsq{}r\PYGZsq{}, \PYGZsq{}d\PYGZsq{}, \PYGZsq{}b\PYGZsq{}, \PYGZsq{}m\PYGZsq{}, \PYGZsq{}z\PYGZsq{}, \PYGZsq{}l\PYGZsq{}\PYGZcb{}}
\end{Verbatim}


\chapter{Set Comprehension}
\label{index:set-comprehension}
\begin{Verbatim}[commandchars=\\\{\}]
\PYG{g+gp}{\PYGZgt{}\PYGZgt{}\PYGZgt{} }\PYG{n}{a} \PYG{o}{=} \PYG{p}{\PYGZob{}}\PYG{n}{x} \PYG{k}{for} \PYG{n}{x} \PYG{o+ow}{in} \PYG{l+s}{\PYGZsq{}}\PYG{l+s}{abracadabra}\PYG{l+s}{\PYGZsq{}} \PYG{k}{if} \PYG{n}{x} \PYG{o+ow}{not} \PYG{o+ow}{in} \PYG{l+s}{\PYGZsq{}}\PYG{l+s}{abc}\PYG{l+s}{\PYGZsq{}}\PYG{p}{\PYGZcb{}}
\PYG{g+gp}{\PYGZgt{}\PYGZgt{}\PYGZgt{} }\PYG{n}{a}
\PYG{g+go}{\PYGZob{}\PYGZsq{}r\PYGZsq{}, \PYGZsq{}d\PYGZsq{}\PYGZcb{}}
\end{Verbatim}


\chapter{Schleifentechniken für Dictionaries}
\label{index:schleifentechniken-fur-dictionaries}
\begin{Verbatim}[commandchars=\\\{\}]
\PYG{g+gp}{\PYGZgt{}\PYGZgt{}\PYGZgt{} }\PYG{n}{knights} \PYG{o}{=} \PYG{p}{\PYGZob{}}\PYG{l+s}{\PYGZsq{}}\PYG{l+s}{Gallahad}\PYG{l+s}{\PYGZsq{}}\PYG{p}{:} \PYG{l+s}{\PYGZsq{}}\PYG{l+s}{der Reine}\PYG{l+s}{\PYGZsq{}}\PYG{p}{,} \PYG{l+s}{\PYGZsq{}}\PYG{l+s}{Robin}\PYG{l+s}{\PYGZsq{}}\PYG{p}{:} \PYG{l+s}{\PYGZsq{}}\PYG{l+s}{der Mutige}\PYG{l+s}{\PYGZsq{}}\PYG{p}{\PYGZcb{}}
\end{Verbatim}

\begin{Verbatim}[commandchars=\\\{\}]
\PYG{g+gp}{\PYGZgt{}\PYGZgt{}\PYGZgt{} }\PYG{k}{for} \PYG{n}{k} \PYG{o+ow}{in} \PYG{n}{knights}\PYG{p}{:}
\PYG{g+gp}{\PYGZgt{}\PYGZgt{}\PYGZgt{} }    \PYG{k}{print} \PYG{p}{(}\PYG{n}{k}\PYG{p}{,} \PYG{n}{knights}\PYG{p}{[}\PYG{n}{k}\PYG{p}{]}\PYG{p}{)}
\PYG{g+gp}{...}
\PYG{g+go}{Gallahad der Reine}
\PYG{g+go}{Robin der Mutige}
\end{Verbatim}

\begin{Verbatim}[commandchars=\\\{\}]
\PYG{g+gp}{\PYGZgt{}\PYGZgt{}\PYGZgt{} }\PYG{k}{for} \PYG{n}{k}\PYG{p}{,} \PYG{n}{v} \PYG{o+ow}{in} \PYG{n}{knights}\PYG{o}{.}\PYG{n}{items}\PYG{p}{(}\PYG{p}{)}\PYG{p}{:}
\PYG{g+gp}{... }    \PYG{k}{print}\PYG{p}{(}\PYG{n}{k}\PYG{p}{,} \PYG{n}{v}\PYG{p}{)}
\PYG{g+gp}{...}
\PYG{g+go}{Gallahad der Reine}
\PYG{g+go}{Robin der Mutige}
\end{Verbatim}


\chapter{Schleifentechniken für Sequenzen}
\label{index:schleifentechniken-fur-sequenzen}
\begin{Verbatim}[commandchars=\\\{\}]
\PYG{g+gp}{\PYGZgt{}\PYGZgt{}\PYGZgt{} }\PYG{n}{i} \PYG{o}{=} \PYG{l+m+mi}{0}
\PYG{g+gp}{\PYGZgt{}\PYGZgt{}\PYGZgt{} }\PYG{k}{for} \PYG{n}{v} \PYG{o+ow}{in} \PYG{p}{[}\PYG{l+s}{\PYGZsq{}}\PYG{l+s}{tic}\PYG{l+s}{\PYGZsq{}}\PYG{p}{,} \PYG{l+s}{\PYGZsq{}}\PYG{l+s}{tac}\PYG{l+s}{\PYGZsq{}}\PYG{p}{,} \PYG{l+s}{\PYGZsq{}}\PYG{l+s}{toe}\PYG{l+s}{\PYGZsq{}}\PYG{p}{]}\PYG{p}{:}
\PYG{g+gp}{... }    \PYG{k}{print}\PYG{p}{(}\PYG{n}{i}\PYG{p}{,} \PYG{n}{v}\PYG{p}{)}
\PYG{g+gp}{... }    \PYG{n}{i} \PYG{o}{+}\PYG{o}{=} \PYG{l+m+mi}{1}
\PYG{g+gp}{...}
\PYG{g+go}{0 tic}
\PYG{g+go}{1 tac}
\PYG{g+go}{2 toe}
\end{Verbatim}

\begin{Verbatim}[commandchars=\\\{\}]
\PYG{g+gp}{\PYGZgt{}\PYGZgt{}\PYGZgt{} }\PYG{k}{for} \PYG{n}{i}\PYG{p}{,} \PYG{n}{v} \PYG{o+ow}{in} \PYG{n+nb}{enumerate}\PYG{p}{(}\PYG{p}{[}\PYG{l+s}{\PYGZsq{}}\PYG{l+s}{tic}\PYG{l+s}{\PYGZsq{}}\PYG{p}{,} \PYG{l+s}{\PYGZsq{}}\PYG{l+s}{tac}\PYG{l+s}{\PYGZsq{}}\PYG{p}{,} \PYG{l+s}{\PYGZsq{}}\PYG{l+s}{toe}\PYG{l+s}{\PYGZsq{}}\PYG{p}{]}\PYG{p}{)}\PYG{p}{:}
\PYG{g+gp}{... }    \PYG{k}{print}\PYG{p}{(}\PYG{n}{i}\PYG{p}{,} \PYG{n}{v}\PYG{p}{)}
\PYG{g+gp}{...}
\PYG{g+go}{0 tic}
\PYG{g+go}{1 tac}
\PYG{g+go}{2 toe}
\end{Verbatim}


\chapter{Module}
\label{index:module}
Datei, die Python-Definitionen und -Anweisungen beinhaltet


\chapter{Module}
\label{index:id10}
\begin{Verbatim}[commandchars=\\\{\}]
\PYG{c}{\PYGZsh{} fibo.py}
\PYG{c}{\PYGZsh{} Fibonacci\PYGZhy{}Zahlen\PYGZhy{}Modul}

\PYG{k}{def} \PYG{n+nf}{fib}\PYG{p}{(}\PYG{n}{n}\PYG{p}{)}\PYG{p}{:}    \PYG{c}{\PYGZsh{} schreibe Fibonacci\PYGZhy{}Folge bis n}
    \PYG{n}{a}\PYG{p}{,} \PYG{n}{b} \PYG{o}{=} \PYG{l+m+mi}{0}\PYG{p}{,} \PYG{l+m+mi}{1}
    \PYG{k}{while} \PYG{n}{b} \PYG{o}{\PYGZlt{}} \PYG{n}{n}\PYG{p}{:}
        \PYG{k}{print}\PYG{p}{(}\PYG{n}{b}\PYG{p}{,} \PYG{n}{end}\PYG{o}{=}\PYG{l+s}{\PYGZsq{}}\PYG{l+s}{ }\PYG{l+s}{\PYGZsq{}}\PYG{p}{)}
        \PYG{n}{a}\PYG{p}{,} \PYG{n}{b} \PYG{o}{=} \PYG{n}{b}\PYG{p}{,} \PYG{n}{a}\PYG{o}{+}\PYG{n}{b}
    \PYG{k}{print}\PYG{p}{(}\PYG{p}{)}

\PYG{k}{def} \PYG{n+nf}{fib2}\PYG{p}{(}\PYG{n}{n}\PYG{p}{)}\PYG{p}{:} \PYG{c}{\PYGZsh{} gib die Fibonacci\PYGZhy{}Folge zurück bis n}
    \PYG{n}{result} \PYG{o}{=} \PYG{p}{[}\PYG{p}{]}
    \PYG{n}{a}\PYG{p}{,} \PYG{n}{b} \PYG{o}{=} \PYG{l+m+mi}{0}\PYG{p}{,} \PYG{l+m+mi}{1}
    \PYG{k}{while} \PYG{n}{b} \PYG{o}{\PYGZlt{}} \PYG{n}{n}\PYG{p}{:}
        \PYG{n}{result}\PYG{o}{.}\PYG{n}{append}\PYG{p}{(}\PYG{n}{b}\PYG{p}{)}
        \PYG{n}{a}\PYG{p}{,} \PYG{n}{b} \PYG{o}{=} \PYG{n}{b}\PYG{p}{,} \PYG{n}{a}\PYG{o}{+}\PYG{n}{b}
    \PYG{k}{return} \PYG{n}{result}
\end{Verbatim}


\chapter{Module}
\label{index:id11}
\begin{Verbatim}[commandchars=\\\{\}]
\PYG{g+gp}{\PYGZgt{}\PYGZgt{}\PYGZgt{} }\PYG{k+kn}{import} \PYG{n+nn}{fibo}
\end{Verbatim}

\begin{Verbatim}[commandchars=\\\{\}]
\PYG{g+gp}{\PYGZgt{}\PYGZgt{}\PYGZgt{} }\PYG{n}{fibo}\PYG{o}{.}\PYG{n}{fib}\PYG{p}{(}\PYG{l+m+mi}{1000}\PYG{p}{)}
\PYG{g+go}{1 1 2 3 5 8 13 21 34 55 89 144 233 377 610 987}
\PYG{g+gp}{\PYGZgt{}\PYGZgt{}\PYGZgt{} }\PYG{n}{fibo}\PYG{o}{.}\PYG{n}{fib2}\PYG{p}{(}\PYG{l+m+mi}{100}\PYG{p}{)}
\PYG{g+go}{[1, 1, 2, 3, 5, 8, 13, 21, 34, 55, 89]}
\PYG{g+gp}{\PYGZgt{}\PYGZgt{}\PYGZgt{} }\PYG{n}{fibo}\PYG{o}{.}\PYG{n}{\PYGZus{}\PYGZus{}name\PYGZus{}\PYGZus{}}
\PYG{g+go}{\PYGZsq{}fibo\PYGZsq{}}
\end{Verbatim}

\begin{Verbatim}[commandchars=\\\{\}]
\PYG{g+gp}{\PYGZgt{}\PYGZgt{}\PYGZgt{} }\PYG{k+kn}{from} \PYG{n+nn}{fibo} \PYG{k+kn}{import} \PYG{n}{fib}\PYG{p}{,} \PYG{n}{fib2}
\PYG{g+gp}{\PYGZgt{}\PYGZgt{}\PYGZgt{} }\PYG{n}{fib}\PYG{p}{(}\PYG{l+m+mi}{500}\PYG{p}{)}
\PYG{g+go}{1 1 2 3 5 8 13 21 34 55 89 144 233 377}
\end{Verbatim}

\begin{Verbatim}[commandchars=\\\{\}]
\PYG{g+gp}{\PYGZgt{}\PYGZgt{}\PYGZgt{} }\PYG{k+kn}{from} \PYG{n+nn}{fibo} \PYG{k+kn}{import} \PYG{o}{*}
\end{Verbatim}


\chapter{Module als Skript aufrufen}
\label{index:module-als-skript-aufrufen}
\begin{Verbatim}[commandchars=\\\{\}]
\PYG{k}{if} \PYG{n}{\PYGZus{}\PYGZus{}name\PYGZus{}\PYGZus{}} \PYG{o}{==} \PYG{l+s}{\PYGZdq{}}\PYG{l+s}{\PYGZus{}\PYGZus{}main\PYGZus{}\PYGZus{}}\PYG{l+s}{\PYGZdq{}}\PYG{p}{:}
    \PYG{k+kn}{import} \PYG{n+nn}{sys}
    \PYG{n}{fib}\PYG{p}{(}\PYG{n+nb}{int}\PYG{p}{(}\PYG{n}{sys}\PYG{o}{.}\PYG{n}{argv}\PYG{p}{[}\PYG{l+m+mi}{1}\PYG{p}{]}\PYG{p}{)}\PYG{p}{)}
\end{Verbatim}

In der Kommandozeile:

\begin{Verbatim}[commandchars=\\\{\}]
\PYGZdl{} python fibo.py 50
1 1 2 3 5 8 13 21 34
\end{Verbatim}

Beim Import:

\begin{Verbatim}[commandchars=\\\{\}]
\PYG{g+gp}{\PYGZgt{}\PYGZgt{}\PYGZgt{} }\PYG{k+kn}{import} \PYG{n+nn}{fibo}
\PYG{g+go}{\PYGZgt{}\PYGZgt{}\PYGZgt{}}
\end{Verbatim}


\chapter{Pakete}
\label{index:pakete}
\begin{Verbatim}[commandchars=\\\{\}]
sound/                             Top\PYGZhy{}level package
         \PYGZus{}\PYGZus{}init\PYGZus{}\PYGZus{}.py               Initialize the sound package
         formats/                  Subpackage for file format conversions
                 \PYGZus{}\PYGZus{}init\PYGZus{}\PYGZus{}.py
                 wavread.py
                 wavwrite.py
                 aiffread.py
                 aiffwrite.py
                 auread.py
                 auwrite.py
                 ...
         effects/                  Subpackage for sound effects
                 \PYGZus{}\PYGZus{}init\PYGZus{}\PYGZus{}.py
                 echo.py
                 surround.py
                 reverse.py
                 ...
         filters/                  Subpackage for filters
                 \PYGZus{}\PYGZus{}init\PYGZus{}\PYGZus{}.py
                 equalizer.py
                 vocoder.py
                 karaoke.py
                 ...
\end{Verbatim}


\chapter{Pakete}
\label{index:id12}
\begin{Verbatim}[commandchars=\\\{\}]
\PYG{g+gp}{\PYGZgt{}\PYGZgt{}\PYGZgt{} }\PYG{k+kn}{import} \PYG{n+nn}{sound.effects.echo}
\PYG{g+gp}{\PYGZgt{}\PYGZgt{}\PYGZgt{} }\PYG{n}{sound}\PYG{o}{.}\PYG{n}{effects}\PYG{o}{.}\PYG{n}{echo}\PYG{o}{.}\PYG{n}{echofilter}\PYG{p}{(}\PYG{n+nb}{input}\PYG{p}{,} \PYG{n}{output}\PYG{p}{,} \PYG{n}{delay}\PYG{o}{=}\PYG{l+m+mf}{0.7}\PYG{p}{,} \PYG{n}{atten}\PYG{o}{=}\PYG{l+m+mi}{4}\PYG{p}{)}
\end{Verbatim}

\begin{Verbatim}[commandchars=\\\{\}]
\PYG{g+gp}{\PYGZgt{}\PYGZgt{}\PYGZgt{} }\PYG{k+kn}{from} \PYG{n+nn}{sound.effects} \PYG{k+kn}{import} \PYG{n}{echo}
\PYG{g+gp}{\PYGZgt{}\PYGZgt{}\PYGZgt{} }\PYG{n}{echo}\PYG{o}{.}\PYG{n}{echofilter}\PYG{p}{(}\PYG{n+nb}{input}\PYG{p}{,} \PYG{n}{output}\PYG{p}{,} \PYG{n}{delay}\PYG{o}{=}\PYG{l+m+mf}{0.7}\PYG{p}{,} \PYG{n}{atten}\PYG{o}{=}\PYG{l+m+mi}{4}\PYG{p}{)}
\end{Verbatim}

\begin{Verbatim}[commandchars=\\\{\}]
\PYG{g+gp}{\PYGZgt{}\PYGZgt{}\PYGZgt{} }\PYG{k+kn}{from} \PYG{n+nn}{sound.effects.echo} \PYG{k+kn}{import} \PYG{n}{echofilter}
\PYG{g+gp}{\PYGZgt{}\PYGZgt{}\PYGZgt{} }\PYG{n}{echofilter}\PYG{p}{(}\PYG{n+nb}{input}\PYG{p}{,} \PYG{n}{output}\PYG{p}{,} \PYG{n}{delay}\PYG{o}{=}\PYG{l+m+mf}{0.7}\PYG{p}{,} \PYG{n}{atten}\PYG{o}{=}\PYG{l+m+mi}{4}\PYG{p}{)}
\end{Verbatim}


\chapter{Referenzen innerhalb des Paketes}
\label{index:referenzen-innerhalb-des-paketes}
\begin{Verbatim}[commandchars=\\\{\}]
\PYG{k+kn}{from} \PYG{n+nn}{.} \PYG{k+kn}{import} \PYG{n}{echo}
\PYG{k+kn}{from} \PYG{n+nn}{..} \PYG{k+kn}{import} \PYG{n}{formats}
\PYG{k+kn}{from} \PYG{n+nn}{..filters} \PYG{k+kn}{import} \PYG{n}{equalizer}
\end{Verbatim}


\chapter{Fehler und Ausnahmen}
\label{index:fehler-und-ausnahmen}
Syntaxfehler

\begin{Verbatim}[commandchars=\\\{\}]
\PYG{g+gp}{\PYGZgt{}\PYGZgt{}\PYGZgt{} }\PYG{k}{while} \PYG{n+nb+bp}{True} \PYG{k}{print}\PYG{p}{(}\PYG{l+s}{\PYGZsq{}}\PYG{l+s}{Hallo Welt}\PYG{l+s}{\PYGZsq{}}\PYG{p}{)}
\PYG{g+go}{  File \PYGZdq{}\PYGZlt{}stdin\PYGZgt{}\PYGZdq{}, line 1, in ?}
\PYG{g+go}{    while True print(\PYGZsq{}Hallo Welt\PYGZsq{})}
\PYG{g+go}{                   \PYGZca{}}
\PYG{g+go}{SyntaxError: invalid syntax}
\end{Verbatim}


\chapter{Fehler und Ausnahmen}
\label{index:id13}
Ausnahmen

\begin{Verbatim}[commandchars=\\\{\}]
\PYG{g+gp}{\PYGZgt{}\PYGZgt{}\PYGZgt{} }\PYG{l+m+mi}{10} \PYG{o}{*} \PYG{p}{(}\PYG{l+m+mi}{1}\PYG{o}{/}\PYG{l+m+mi}{0}\PYG{p}{)}
\PYG{g+gt}{Traceback (most recent call last):}
  File \PYG{n+nb}{\PYGZdq{}\PYGZlt{}stdin\PYGZgt{}\PYGZdq{}}, line \PYG{l+m}{1}, in \PYG{n}{?}
\PYG{g+gr}{ZeroDivisionError}: \PYG{n}{int division or modulo by zero}
\PYG{g+gp}{\PYGZgt{}\PYGZgt{}\PYGZgt{} }\PYG{l+m+mi}{4} \PYG{o}{+} \PYG{n}{spam}\PYG{o}{*}\PYG{l+m+mi}{3}
\PYG{g+gt}{Traceback (most recent call last):}
  File \PYG{n+nb}{\PYGZdq{}\PYGZlt{}stdin\PYGZgt{}\PYGZdq{}}, line \PYG{l+m}{1}, in \PYG{n}{?}
\PYG{g+gr}{NameError}: \PYG{n}{name \PYGZsq{}spam\PYGZsq{} is not defined}
\PYG{g+gp}{\PYGZgt{}\PYGZgt{}\PYGZgt{} }\PYG{l+s}{\PYGZsq{}}\PYG{l+s}{2}\PYG{l+s}{\PYGZsq{}} \PYG{o}{+} \PYG{l+m+mi}{2}
\PYG{g+gt}{Traceback (most recent call last):}
  File \PYG{n+nb}{\PYGZdq{}\PYGZlt{}stdin\PYGZgt{}\PYGZdq{}}, line \PYG{l+m}{1}, in \PYG{n}{?}
\PYG{g+gr}{TypeError}: \PYG{n}{Can\PYGZsq{}t convert \PYGZsq{}int\PYGZsq{} object to str implicitly}
\end{Verbatim}


\chapter{Klassen}
\label{index:klassen}
\begin{Verbatim}[commandchars=\\\{\}]
class Host:
    id = 42
    def \PYGZus{}\PYGZus{}init\PYGZus{}\PYGZus{}(self, mac)
        self.mac = mac

    def get\PYGZus{}dhcp\PYGZus{}lease():
        pass
\end{Verbatim}


\chapter{Vererbung}
\label{index:vererbung}
\begin{Verbatim}[commandchars=\\\{\}]
class Host:
    id = 42
    def \PYGZus{}\PYGZus{}init\PYGZus{}\PYGZus{}(self, mac)
        self.mac = mac

    def get\PYGZus{}dhcp\PYGZus{}lease():
        pass

class Workstation(Host):
    def install\PYGZus{}updates():
        pass
\end{Verbatim}


\chapter{Mehrfachvererbung}
\label{index:mehrfachvererbung}
\begin{Verbatim}[commandchars=\\\{\}]
class Host:
    id = 42
    def \PYGZus{}\PYGZus{}init\PYGZus{}\PYGZus{}(self, mac)
        self.mac = mac

    def get\PYGZus{}dhcp\PYGZus{}lease():
        pass

class Workstation(Host):
    def install\PYGZus{}updates():
        pass

class Appliance:
    def turn\PYGZus{}off():
        pass

class SmartFridge(Appliance, Host):
    pass
\end{Verbatim}


\chapter{\textless{}EOF\textgreater{}}
\label{index:eof}\begin{itemize}
\item {} 
@danielhepper

\item {} 
\href{https://github.com/dhepper/froscon2016-python-intro}{https://github.com/dhepper/froscon2016-python-intro}

\item {} 
Basierend auf \href{http://py-tutorial-de.readthedocs.io/de/python-3.3/}{http://py-tutorial-de.readthedocs.io/de/python-3.3/}

\item {} 
Lizenz: Apache License, Version 2.0

\end{itemize}



\renewcommand{\indexname}{Stichwortverzeichnis}
\printindex
\end{document}
